\documentclass[12pt]{article}
\usepackage{graphicx}
\setlength{\textwidth}{6.5in}
\setlength{\textheight}{9.5in}
\setlength{\topmargin}{-0.7in}
\setlength{\evensidemargin}{0in}
\setlength{\oddsidemargin}{0in}
\newcommand{\mysection}[1]{
~\\ \noindent
{\bf \normalsize #1}
\vspace{1mm}
}


\begin{document}
\renewcommand\refname{}


\section*{%
Validation and parameter selection for
the GREIT \\ reconstruction algorithm
}

\vspace{-2mm}
\begin{list}{}{\setlength\leftmargin{25mm}}
\item[]\raggedright
 \footnotesize
{\bf
        A Adler$^{1}$
        J H Arnold$^{2}$
        R Bayford$^{3}$
        A Borsic$^{4}$
        B Brown$^{5}$
        P Dixon$^{6}$
        T J C Faes$^{7}$
        I Frerichs$^{8}$
        H Gagnon$^{9}$
        Y G\"arber$^{10}$
        B Grychtol$^{11}$, 
        G Hahn$^{12}$
        W R B Lionheart$^{13}$
        A Malik$^{14}$
        R P Patterson$^{15}$
        J Stocks$^{16}$
        A Tizzard$^{3}$
        N Weiler$^{8}$
        G K Wolf$^{2}$
}
\\
          $^{1}$Carleton University, Ottawa
          $^{2}$Children's Hospital Boston
          $^{3}$Middlesex University, London
          $^{4}$Dartmouth College, Hanover, NH
          $^{5}$University of Sheffield
          $^{6}$Cardinal Health Care, London
          $^{7}$VUMC, Amsterdam
          $^{8}$University of Schleswig-Holstein
          $^{9}$\'Ecole Polytechnique de Montr\'eal
         $^{10}$Dr\"ager Medical, L\"ubeck
         $^{11}$University of Strathclyde
         $^{12}$University of G\"ottingen
         $^{13}$University of Manchester
         $^{14}$Maltron International, Rayleigh
         $^{15}$University of Minnesota
         $^{16}$University College London
\end{list}

\vspace{-5mm}
\small
\mysection{Introduction}

One of the most promising applications of EIT is to monitor the
regional distribution of ventilation in mechanically
ventilated patients.
Based on EIT images, ventilation strategies could be
optimized to hopefully improve gas exchange, recruit
atelectatic areas, minimize regional overdistension and
therefore reduce ventilator induced lung injury.
One key limitation is that
the majority of clinical and experimental EIT research
has been performed with the Sheffield backprojection
reconstruction algorithm developed in the early 90's; this
is an obstacle to interpretation of EIT images because
the reconstructed images are not well characterized. To
address this issue, we are developing a consensus linear
reconstruction algorithm for lung EIT, called GREIT (Graz
consensus Reconstruction algorithm for EIT).  The goal of
GREIT is to develop a consensus on the desired properties
of a linear reconstruction algorithm, and implement an
approach to achieve a good fit to these properties.
This paper describes the GREIT image reconstruction
framework developed, and
current work to choose select values for
algorithm parameter to provide stable and 
accurate clinical images.

\vspace{-3mm}
\mysection{GREIT Image Reconstruction Framework}

The framework developed\cite{GREIT09} consists of: 1)
detailed finite element models of a representative adult
and neonatal thorax; 2) consensus on the performance
figures of merit for EIT image reconstruction; and 3) a
systematic approach to optimize a linear reconstruction
matrix to desired performance measures (Fig.\ \ref{fig:figparams}, Left).
The consensus figures of merit (Fig.\ \ref{fig:figparams}, Centre)
 in agreed order of importance,
are: a) uniform amplitude response (AR), b) small and
uniform position error (PE), c) small ringing artefacts (RNG),
d) uniform resolution (RES), e) limited shape deformation (SD),
and f) high resolution (RES). 
As shown in Fig.\ \ref{fig:figparams} (Right), GREIT is able to 
achieve more uniform spatial resolution with low ringing at the
expense of a decrease in spatial resolution near the boundary.
\vspace{-4mm}
\begin{figure}[htp]
\centering
\includegraphics[height=45mm]{fig-params.eps}
\vspace{-6mm}
\caption{%
\small
{\em Left:} GREIT formulation. Each training data point
    ${\bf x}_t^{(k)}$ corresponds to a {\em desired response},
$\tilde{\bf x}^{(k)}$, and a reconstructed image,
  $\hat{\bf x}^{(k)}$;
{\em Centre:} Figures of merit calculated from $\hat{\bf x}^{(k)}$;
{\em Right:} Example reconstructed images for three different
targets from
Sheffield Backprojection (SBP), a NOSER type Gauss-Newton solver (GN),
and GREIT (GR).
}
\label{fig:figparams}
\end{figure}

\mysection{Selection of parameter values}

Several parameters in the GREIT framework need to be
determined based on a careful evaluation strategy:
1) training noise levels,
2) model accuracy,
3) the use (or not) of normalized difference imaging,
4) parameter settings for movement compensation, and
5) assumptions about the reference conductivity of the thorax.

In this section, we illustrate the effect of one parameter
with a significant impact on reconstructed images: the
reference conductivity of the thorax.
While most time-difference EIT
images of the lungs assume a homogeneous conductivity 
distribution, this is clearly not valid for the thorax. 
The homogeneous assumption tests to ``squeeze'' the lung
images together. Fig.\ \ref{fig:figbackgnd} shows images
of tidal volume in a pig lung injury model (data from 
\cite{frerichs03}) and the effect of the reference
conductivity parameter. Interestingly, the effect of changing
the reference conductivity is different for 2D and 
3D models, which underlines the importance of using accurate
FEM models.

\vspace{-2mm}
\begin{figure}[htp]
\centering
\includegraphics[height=50mm]{fig-backgnd.eps}
 \setlength{\unitlength}{1cm}
 \put( -13.40, 4.50){ \framebox{\textbf{\textsf{A}}}}
 \put( -10.80, 4.50){ \framebox{\textbf{\textsf{B}}}}
 \put(  -8.20, 4.50){ \framebox{\textbf{\textsf{C}}}}
 \put(  -5.60, 4.50){ \framebox{\textbf{\textsf{D}}}}
 \put(  -3.00, 4.50){ \framebox{\textbf{\textsf{E}}}}

\vspace{-4mm}
\caption{%
\small
Images of tidal volume in lung-injured piglet during
a recruitment manouvre at P{\footnotesize EEP=}10cm$\rm H_{2}O$
(data from \cite{frerichs03}) using different reconstruction
algorithm parameter values. All algorithms are normalized to have
the same noise performance as Sheffield Backprojection.
{\em Top Row:} Uniform reference conductivity
{\em Bottom Row:} Central region conductivity of 0.3$\times$reference.
{\em A:} Sheffield Backprojection
{\em B:} GREIT (3D model, normalized difference)
{\em C:} GREIT (3D model, difference)
{\em D:} GREIT (2D model, normalized difference)
{\em E:} GREIT (2D model, difference)
}
\label{fig:figbackgnd}
\end{figure}

\vspace{-3mm}
Clinical and experimental evaluation of GREIT is being
performed in order to choose appropriate algorithm values,
and evaluate image reconstruction performance with 
clinical data collected from ventilated subjects.
A collection of clinical and experimental data has been assembled from
animal and human studies in neonates, children and adult patients in
Canada, Germany, UK and USA.  Evaluation will consist
of reconstructing images with variants of GREIT, and
asking experts to score the images.  One concern with any
new medical imaging algorithm is its performance in the
presence of data artefacts. With experience, such artefacts
can be identified in the images;  thus, another goal
of GREIT evaluation is to develop a set of heuristics
 to detect such data errors.


\vspace{-4mm}
\mysection{Discussion}

The goal of GREIT is to develop a well characterized 2D
linear difference EIT algorithm for lung imaging. We seek to develop
a robust algorithm which incorporates the key developments
in EIT lung image reconstruction. Our experience to date
with GREIT shows improvements in resolution of image features
as well as reduced image artefacts when compared with an
implementation of the Sheffield backprojection
algorithm. In order to facilitate use of GREIT,
all software
and data to implement and test the algorithm have been
made available on the internet at eidors.org/GREIT under
an open source license which allows free research and
commercial use.

\vspace{-4mm}
\mysection{References}
\begin{thebibliography}{9}
\setlength{\itemsep}{-2mm}
\vspace{-1.8cm}
\bibitem{GREIT09} 
Adler A  Arnold JA Bayford R {\em etal} (2009)
In Press {\em Physiol Meas}

\bibitem{frerichs03}
Frerichs I Dargavillle PA Dudykevych T Rimensberger PM (2003) 
{\em Int Care Med} 29:2312-6

\end{thebibliography}


\end{document}
