% Main file for the GREIT Algorithm
% $Id$
\documentclass[12pt]{iopart}
\usepackage{graphicx}
 \usepackage{amssymb}
 \usepackage{amsbsy}
\newcommand{\vB}{\mbox{$\mathbf{v}$}}
\newcommand{\xB}{\mbox{$\mathbf{x}$}}
\newcommand{\xH}{\mbox{$\mathbf{\hat x}$}}
\newcommand{\xT}{\mbox{$\mathbf{\tilde x}$}}
\newcommand{\XT}{\mbox{$\mathbf{\tilde X}$}}
\newcommand{\nB}{\mbox{$\mathbf{n}$}}
\newcommand{\yB}{\mbox{$\mathbf{y}$}}
\newcommand{\wB}{\mbox{$\mathbf{w}$}}
\newcommand{\AB}{\mbox{$\mathbf{A}$}}
\newcommand{\BB}{\mbox{$\mathbf{B}$}}
\newcommand{\RB}{\mbox{$\mathbf{R}$}}
\newcommand{\IB}{\mbox{$\mathbf{I}$}}
\newcommand{\JB}{\mbox{$\mathbf{J}$}}
\renewcommand{\PB}{\mbox{$\mathbf{P}$}}
\newcommand{\VB}{\mbox{$\mathbf{V}$}}
\newcommand{\WB}{\mbox{$\mathbf{W}$}}
\newcommand{\XB}{\mbox{$\mathbf{X}$}}
\newcommand{\YB}{\mbox{$\mathbf{Y}$}}
% Use boldsymbol when using amssymb
 \newcommand{\SG}{\mbox{${\boldsymbol \Sigma}$}}
 \newcommand{\TG}{\mbox{${\boldsymbol \Theta}$}}
 \newcommand{\sG}{\mbox{${\boldsymbol \sigma}$}}
%\newcommand{\SG}{\mbox{${\mathbf \Sigma}$}}
%\newcommand{\TG}{\mbox{${\mathbf \Theta}$}}
%\newcommand{\sG}{\mbox{${\mathbf \sigma}$}}
\newcommand{\SNR}{\mbox{\small $\mathrm{SNR }$}}
\newcommand{\NF}{\mbox{\small $\mathrm{NF }$}}
\newcommand{\EIT}{\mbox{\small $\mathit{EIT }$}}
\begin{document}

\title[GREIT: linear EIT image reconstruction]{%
GREIT: a unified approach to 2D linear EIT reconstruction of
       lung images
\\
{\small \tt DRAFT: $Date$}
}

\author{Andy Adler$^{1}$,
        John Arnold$^{2}$,
        Richard Bayford$^{3}$,
        Andrea Borsic$^{4}$,
        Brian Brown$^{5}$,
        Paul Dixon$^{6}$,
        Theo J.C. Faes$^{7}$,
        In\'ez Frerichs$^{8}$,
        Herv\'e Gagnon$^{9}$,
        Yvo G\"arber$^{10}$,
        Bart\l{}omiej Grychtol$^{11}$, 
        G\"unter Hahn$^{12}$,
        William R B Lionheart$^{13}$,
        Anjum Malik$^{14}$,
        Janet Stocks$^{15}$,
        Andrew Tizzard$^{3}$,
        Norbert Weiler$^{8}$,
        Gerhard Wolf$^{2}$%
       }

\address{ $^{1}$Systems and Computer Engineering,
                Carleton University, Ottawa, Canada}
\address{ $^{2}$Division of Critical Care Medicine, Department of Anesthesia,
                Children's Hospital Boston, Harvard Medical School,
                Boston, MA, USA}
\address{ $^{3}$School of Health and Social Sciences,
                Middlesex University, London, UK}
\address{ $^{4}$School of Engineering, 
                Dartmouth College, Hanover, NH, USA}
\address{ $^{5}$Medical Physics, University of Sheffield, UK}
\address{ $^{6}$Cardinal Health Care, London, UK}
\address{ $^{7}$Department of Physics and Medical Technology,
                V.U. university medical center, Amsterdam, Netherlands}
\address{ $^{8}$Department of Anaesthesiology and Intensive Care Medicine,
                University of Schleswig-Holstein, Kiel, Germany}
\address{ $^{9}$D\'epartement de g\'enie \'electrique,
                \'Ecole Polytechnique de Montr\'eal, Canada}
\address{$^{10}$Dr\"ager Medical, L\"ubeck, Germany}
\address{$^{11}$University of Strathclyde, Glasgow, UK}
\address{$^{12}$Department of Anaesthesiological Research,
                University of G\"ottingen, Germany}
\address{$^{13}$School of Mathematics, University of Manchester, UK}
\address{$^{14}$Maltron International, Rayleigh, UK}
\address{$^{15}$Institute of Child Health,
                University College London, UK}



\begin{abstract}
Recently, Electrical Impedance Tomography (EIT) has begun to see a
significant clinical interest for the application of monitoring
mechanically ventilated patients.
EIT has clinical promise to help better manage these patients,
because it is able to non-invasively
provide a continuous image of the distribution of ventilation. 
However, most clinical and physiological research in lung EIT
is done using older and proprietary algorithms; this is
an obstacle to interpretation of EIT data because the
reconstructed images are not well characterized.
To address this issue, we are developing a
consensus linear reconstruction algorithm for lung EIT,
called GREIT (Graz consensus Reconstruction algorithm for EIT).
This paper describes the unified approach to 
linear image reconstruction developed for GREIT,
and the tools and evaluation methodology for
clinical and experimental testing of the algorithm.
The framework for the linear reconstruction algorithm
consists of:
1) detailed finite element models (FEM) of a representative
 adult and neonatal thorax;
2) consensus on the performance figures of merit for
 EIT image reconstruction, which are:
a) uniform amplitude response,
b) low and uniform reconstructed position error,
c) low, uniform and circular small point spread function,
d) good noise performance,
e) low sensitivity to electrode and boundary movement,
f) low artefact amplitude on clinical and experimental data;
and
3) A systematic approach to optimize a linear reconstruction
 matrix to desired performance figures.
This approach represents the consensus of a large and representative
group of experts in EIT algorithm and clinical applications.
All software and data to implement and test GREIT has be
made available under an open source license which allows free
research and commercial use. These tools will be used in
the clinical and experimental validation stage of GREIT development.
\end{abstract}

\noindent{\it Keywords\/}:
Electrical Impedance Tomography,
Lung Function Imaging,
Image Reconstruction,

\section{Introduction}
Electrical Impedance Tomography (EIT) measures conductivity
changes within a body from current stimulation and voltage
measurement on the body surface. One of the most promising
applications of EIT is for measuring the lungs, since these
are large organs which undergo large changes in conductivity
during normal functioning. Indeed, lung function measurement
was among the first physiological applications of this technology.
(Barber and Brown 1984).
Some imaging modalities can measure ventilation,
such as MRI (with hyperpolarized He) and PET. However, commonly
used imaging modalities (radiography, CT, MRI) do not assess
ventilation directly.
EIT is unique in that it
is able to non-invasively and continuously monitor the distribution of 
ventilation.  Based on these advantages, there is significant
interest in EIT to 
monitor patients with respiratory compromise.

One limitation is that most clinical and physiological research
on lung EIT is being done using proprietary variants of
older image reconstruction algorithms, such as the backprojection
algorithm as implemented
in the Sheffield (Brown and Seagar, 1987)
or G\"ottingen (Hahn \etal, 2001) EIT systems.
The algorithms, although very successful for the time,
   do not incorporate advances that have been made over the last
   20 years.
This is an obstacle to clinical use of EIT because
these algorithms have numerous specific issues, such
   as spatial non-uniformity in image amplitude, position
   and resolution, which make interpretation of regional
   ventilation difficult and error prone.
One specific issue, is that 
artefacts can be incorrectly interpreted (i.e. movement of
   the skin can appear as changes in the lung regions),
   resulting in incorrect interpretations.
Many other approaches to reconstruct EIT images have been
proposed, incorporating many advances (Lionheart, 2004).
Such approaches
have not been widely used clinically and experimentally
because there was a lack of a agreement on which
approaches were best (and whether they could be combined).

To address this issue, we are developing a
consensus linear reconstruction algorithm for lung EIT,
called GREIT (Graz consensus Reconstruction algorithm for EIT),
since early discussions took place at the 2007 ICEBI conference
in Graz, Austria (ref). Our aim is to develop a standard which
has broad agreement from experts in the mathematical,
engineering, physiological, and clinical EIT communities.
Such an approach is feasible because there is general
consensus amongst experts in EIT image processing on
the ``ingredients'' that should
be part of a robust and high performance linear algorithm
for 2D EIT of the lungs.
This paper describes the unified approach to 
linear image reconstruction developed for GREIT,
and the tools and evaluation methodology for
clinical and experimental testing of the algorithm.
The framework for the linear reconstruction algorithm
consists of:
1) detailed finite element models of a representative
 adult and neonatal thorax;
2) consensus on the performance figures of merit for
 EIT image reconstruction, which are:
a) uniform amplitude response; b) low and uniform reconstructed position error; c) low, uniform and circular small point spread function; d) good noise performance; e) low sensitivity to electrode and boundary movement; f) low artefact amplitude on clinical and experimental data;
and
3) a systematic approach to optimize a linear reconstruction
 matrix to desired performance figures.

The current work is limited to the reconstruction algorithm.
At this stage, we do not propose calibration tests, data
 formats or phantoms, standards
for image interpretation or EIT based lung parameters; 
we do not feel there is sufficient experience yet to reach
consensus in these areas.
It is important to clarify that there is no financial
goal to the development of this algorithm.
All developed algorithms, software
models and simulation and experimental test data used
in this algorithm have been
made available under an open source license as part of
the open source EIDORS distribution (Adler and Lionheart, 2006);
this license permits
royalty free use in both research and commercial applications.


The following specifications have been defined for GREIT:
\begin{itemize}
\item
 single ring electrode
configurations with Sheffield-type EIT systems, using
      adjacent current injection and measurement.
\item
 linear (real-time) reconstruction of a 2D conductivity
change image, based on a 3D forward model.
\item
 quantitative reconstructions for which units can
  be assigned to EIT images.
\item
 settings for all parameters:
     any tunable parameters must have assigned
     values in the recommended algorithm.
\item
 published reconstruction matrices for
      a $32\times 32$ pixel array
      for a single ring of $16$, $12$ and $8$ 
      electrodes, for the shapes:
   a) neonatal chest, 
   b) adult chest (for perhaps several body shapes), and 
   c) cylindrical tank phantoms.
 For other shapes and electrode configurations,
   reconstruction matrices may be calculated from the
   provided source code.
\end{itemize}

These limitations represent, in our opinion, the limits
to areas sufficiently well understood in the EIT community
to reach consensus. We do not suggest that such limits are ideal;
in fact, we actively encourage work to overcome them,
such as planar placement of electrodes and the consequent 2D images. 

\section{Methods: performance figures of merit}
\label{sec:figmerit}

In this section, we elaborate on a set of criteria which 
characterize the performance of an ideal reconstruction
algorithm. Clearly, there is an inherent trade-off between
the measures, such that it is not possible to 
simultaneously optimize all measures. Instead, we proceed
as follows. First, we establish the desired performance measures
of an ideal algorithm; subsequently, 
we describe a unified methodology to calculate a reconstruction
matrix which is an optimized compromise of criteria
weighted by the consensus importance of each.  

We consider an EIT system using the following notation.
$n_E$ electrodes applied are to a body, and use
sequential current stimulation with parallel voltage
measurement. Using these electrodes, $n_E$ current stimulation
patterns are sequentially applied and $n_V$ differential
measurements are made for each stimulation.  For an adjacent drive
EIT system, voltages are typically not measured at driven
electrodes, and $n_V = n_E - 3$.  Each data frame measures
a vector, $\vB\in\mathbb{R}^{n_M}$, of $n_M= n_E n_V$ data points
(some of which are redundant if the medium is not changing).
Difference EIT calculates difference data $\yB$, ($[\yB]_i =
[\vB]_i - [\vB_r]_i$; or the normalized difference data $[\yB]_i
= ([\vB]_i - [\vB_r]_i)/[\vB_r]_i)$, where $\vB_r$ is a reference
set of measurements representing the background conductivity
distribution, $\sG_r$. To improve its precision,
$\vB_r$ is typically averaged over many data frames; 
such ensemble averaging reduces random noise, and we
 assume that $\vB_r$ is noise free.



\begin{figure}[bhtp]
\begin{center}
\includegraphics[width=0.8\textwidth,
                 bb=0.0in 1.7in 10in 6.0in,page=1]
                {figures/fig-perf-params.pdf}
\caption{ \label{fig:PerfFigures}
Performance figures of merit for evaluation of GREIT images.
Based on a reconstructed image ($\xH$, {\em left}),
an image ($\xH_q$, {\em centre})
is constructed of all image pixels which exceed
 $\frac{1}{4}$ the maximum amplitude. From these images,
figures of merit {\em right} are calculated as described.
}
\end{center}
\end{figure}

An outline of the calculation of performance figures
of merit is given in figure \ref{fig:PerfFigures}.
Given a vector of EIT difference or normalized difference
data, $\yB$ (length $n_E n_V$), we calculate a 
reconstructed EIT image $\xH = \RB \yB$ based on
an EIT linear reconstruction algorithm represented as
a matrix $\RB$. $\xH$ is a column vector representing
the $32\times 32$ pixel grid of the images.

We define figures of merit based on small ``point''
conductivity changes, where small indicates a diameter
of less than 5\% of the medium diameter, and are thus
much smaller than the inherent resolution of EIT with
16 electrodes.
To evaluate this image, we first calculate
a $\frac{1}{4}$--amplitude set, $\xH_q$ which
contains all image pixels $[\xH]_i$ greater
than $\frac{1}{4}$ of the maximum amplitude:
\begin{equation}
[\xH_q]_i = \left\{ \begin{array}{ll}
    1 & [\xH]_i  \geq \mbox{$ \frac{1}{4}\mathrm{max}~\xH$ } \\
    0 & \mbox{otherwise}.
\end{array} \right.
\end{equation}
For targets
with a conductivity decrease, the definition
of the $\frac{1}{4}$--amplitude set is inverted.
The centre of gravity (CoG) of $\xH$ and $\xH_q$ are
calculated; the distance from the CoG to the 
medium centre is then calculated, as $r_t$ and $r_q$
respectively.

Based on images of point targets, we define the following
figures of merit:
amplitude response,
position error,
resolution,
shape deformation, and
ringing. 

\begin{itemize}

\item
{\bf Amplitude Response (AR)}
measures the ratio of image pixel amplitudes in the
target to that in the reconstructed image.
For a spherical target of volume, $V_t$, in the electrode
plane with
conductivity $\sigma_t$ and homogeneous reference
conductivity $\sigma_r$
\begin{equation}
\mathrm{AR} =  \frac{
    \sum_k [\xH]_k
  }{
    V_t \frac{\Delta\sigma}{\sigma_r}
  }
\end{equation}
where $\Delta\sigma = \sigma_t - \sigma_r$,

AR allows units to be assigned to the 
reconstructed conductivities. For this, we 
require that reconstruction matrices, $\RB$, be
scaled such that AR$=1$ for 
a small spherical target with
 $\Delta\sigma_t/\sigma_r \approx 1$
in the centre of the medium.
With this normalization, reconstructed EIT images have
units as follows:
   \begin{itemize}
   \item {\em Normalized Difference EIT:}
in this case, measurements $\yB$ are unitless,
and the EIT image will be in units of fractional conductivity
change, $\Delta\sigma/\sigma_r$.
   \item {\em Difference EIT:}
in this case, measurements $\yB$ must be in units of
transfer conductance ($\Omega^{-1}$), and the EIT image
will be in units of conductivity change (in $\Omega^{-1}\cdot$m)
scaled by the ratio or the measurement geometry to
the FEM geometry.
% for a box of A, L, admittance = A sigma / L
% if we scale it by s, admittance = s^2 sigma / sL = s admittance_0
% so admittance scales by the model to real scale ratio.
   \end{itemize}


\hspace{5mm}
{\em Desired behaviour:}
AR should be constant for any target position. We consider
constant
AR to be the most important figure of merit. Without
constant amplitude response, the same volume of air in different parts
of the lung will contribute differently to the image, introducing
serious difficulties in image interpretation.

\item
{\bf Position Error (PE):}
measures the extent to which reconstructed images faithfully
represent the position of the image target. Based on the
target position, $r_t$, and the CoG of $\xH_q$, $r_q$, we define:
\begin{equation}
\mathrm{PE} = r_t - r_q
\end{equation}
where positive values of PE indicate reconstructed images
``pushed'' to the medium centre. This measure is similar
to that of Adler and Guardo (1996).

\hspace{5mm}
{\em Desired behaviour:}
PE should be small and show small variability for
targets at different radial positions. We consider
small and constant PE to be the second most important
figure of merit. If PE is variable, interpretation
of a distribution of air in the lungs becomes unreliable.
Sheffield backprojection has large PE near the
electrodes; this has resulted in cases where changes at
the electrodes are misinterpreted as being inside the body.

\item
{\bf Resolution (RES):}
measures the size of reconstructed targets as a fraction
of the medium; this is
equivalent to a measure of point spread function (PSF) size.
\begin{equation}
\mathrm{RES} = \sqrt{
 \frac{ A_q }
      { A_0 }
 }
\end{equation}
where $A_q =  \sum_k [\xH_q]_k$, the number
of pixels in $\xH_q$ and 
$A_0$ is the area (in pixels) of
the entire reconstructed medium. The square root is used 
so that RES measures radius ratios rather than area ratios.
This measure is modelled on the one proposed by
Wheeler \etal (2002).

\hspace{5mm}
{\em Desired behaviour:}
RES should be uniform and small, in order to 
more accurately represent shape of the target conductivity
distribution. We consider uniformity of RES to be the
fourth most important figure of merit, but small size
of RES to be much less important (seventh). A nonuniform
RES can result in an incorrect reconstructed 
position of a larger target. Low RES serves primarily
to distinguish nearby targets. Since EIT is understood to
be a low resolution medium, this application is less important.

\item
{\bf Shape Deformation (SD):}
Reconstruction algorithms typically create circular
images for targets in the centre, but often display 
strange shaped artefacts for targets near the medium
boundary. Backprojection tends to show ``streak''
artefacts in this case.
SD measured the fraction of the reconstructed
$\frac{1}{4}$--Amplitude set which does not
fit within a circle.
\begin{equation}
\mathrm{SD} = \sum_{k\not\in C} [\xH_q]_k / 
              \sum_{k} [\xH_q]_k
\end{equation}
where $C$ is a circle centred at the CoG of $\xH_q$
with an equivalent area to $\xH_q$. This figure
similar to that of Oh \etal (2007).

\hspace{5mm}
{\em Desired behaviour:}
SD should be low and uniform. Large SD may result in
incorrect interpretation of images, although this
effect is less important than other artefacts. We 
consider SD to be the fifth most important figure 
of merit.

\item
{\bf Ringing (RNG):}
measures whether reconstructed images show
areas of oposite sign surrounding the main 
reconstructed target area. Fig. \ref{fig:PerfFigures}
shows an arc of ringing around the image centre.
Overshoot is typical of linear filters; for second-order
systems the effect may also be called ``overshoot''.
RNG measures the ratio of image amplitude of
the opposite sign outside circle $C$ to image
amplitude within $C$.
\begin{equation}
\mathrm{RNG} = \sum_{k\not\in C \& [\xH]_i < 0} [\xH]_k / 
               \sum_{k\in C}                    [\xH]_k 
\end{equation}

\hspace{5mm}
{\em Desired behaviour:}
RNG should be low and uniform. Overshoot may
easily result in incorrect interpretation. For
example, the negative ringing between non-conductive
lungs will produce a conductive pattern which 
may be misinterpreted as a heart.
We consider RNG to be the third most important figure 
of merit.

\begin{figure}[bhtp]
\begin{center}
\includegraphics[width=0.4\textwidth,
                 bb=1.5in 2.0in 6.5in 6.0in,page=1]
                {figures/fig-noise-fig.pdf}
\caption{ \label{fig:noise_fig}
Schematic representation of the Noise Figure (NF)
parameter. NF represents the amplification of noise through
the reconstruction process as the ratio of SNR$_x$ to SNR$_y$.
}
\end{center}
\end{figure}

\item
{\bf Noise Amplification (NF):}
measures the extent to which random
measurement noise is amplified
in the reconstructed images. Noise amplification
is measured by the noise figure (NF) as
illustrated in Fig. \ref{fig:noise_fig}, 
based on the definition by of Adler and Guardo (1996).
The NF is the ratio of the output to input
 signal to noise ratio (SNR) at the for a filter.
We define SNR=
$\frac{
   \mbox{mean \em signal}
      }{
   \mbox{std \em noise}
     }
$ 
in terms of image amplitude,
rather than in terms of image energy, since linear
image reconstruction tends to conserve  amplitude
rather than energy. Thus
\begin{equation}
\mathrm{NF} = \frac{
   E[ \mathrm{mean}~\xH_t ] 
         /
   E[ \mathrm{std}~\xH_t ]
}{
   E[ \mathrm{mean}~\yB_t ] 
         /
   E[ \mathrm{std}~\yB_t ]
}
\end{equation}
For regularized algorithms, the value of NF is normally
set by selection of the value of 
a hyperparameter. For GREIT, the value of NF is set by the weighting
associated with the training noise. There is an inherent
trade-off between good noise performance and fidelity to
the other figures of merit.

\hspace{5mm}
{\em Desired behaviour:}
NF should be low. Ideally NF should be tuned to the noise
level present in the EIT hardware used.
Several studies have looked into strategies to select
appropriate noise performance (eg. Graham and Adler, 2006).
At this time, we do not have consensus on the best
way to choose NF. Instead, we note that noise performance
of Sheffield Backprojection (NF=$0.5$ in the medium center)
has generally been considered to be satisfactory.
Therefore, as an iterim measure, we recommend NF=$0.5$
for GREIT algorithms.

\end{itemize}

\section{Methods: Reconstruction algorithm framework}

This section describes the GREIT framework to calculate
a linear image reconstruction matrix which is optimized
to a set of performance requirements. This approach
is based on experience with a large number of 
linear regularized EIT reconstruction approaches
(such as
Adler and Guardo 1996,
Adler and Lionheart 2006,
Barber and Brown 1988,
Cheney \etal 1990,
Cohen-Bacrie \etal 1997,
Lionheart \etal 2005,
Polydorides and Lionheart, 2002
Soleimani \etal 2006
Vauhkonen \etal 1998).
The key difference is that this framework
is based on a set of performance requirements, whereas
previous reconstruction algorithms are based directly on
the underlying mathematical models and do not express
the performance requirements explicitly.
The value of this approach is that it allows
performance requirements (established in our case
via consensus) to be directly encoded into the
reconstruction algorithm.

The body under investigation is modelled using a finite element
model (FEM) which discretizes the conductivity onto $n_N$
piecewise smooth elements, represented by a vector
$\sG\in\mathbb{R}^{n_N}$ (To clarify symbols,
$\sG$ represents a vector conductivity distribution, while
$\sigma$ is the standard deviation).
Again, difference EIT calculates a vector of
conductivity change, $\xB = \sG - \sG_r$ between the present,
$\sG$, and the reference, $\sG_r$,
conductivity distributions.

Image reconstruction in EIT is a non-linear problem; however,
linearized approximations have proved to be very useful.
This is largely because the noise levels present in clinical
and experimental EIT data do not typically have sufficient
accuracy to show stable results with iterative algorithms.
Linear algorithms also have the benefit of producing
images in which the effects
of data artefacts can be more readily identified. Finally,
linear reconstruction can be implemented as a fast matrix
multiplication. We represent linear EIT image reconstruction
as a matrix, $\RB\in\mathbb{R}^{n_N\times n_M}$ which
maps measurements $\yB$ to a reconstructed image $\xH$:
\begin{equation} 
\label{reconst_eqn}
   \xH = \RB \yB
\end{equation} 
 

\subsection{Sheffield Backprojection}

The backprojection algorithm was developed by 
Barber and Brown (1983) for the original 
Sheffield EIT system, and  has seen numerous 
variants and improvements. While the original
technique was based on an analogy to the backprojection
algorithm for CT, this model was clearly inadequate, because
diffuse electric current propagation is very different
to that of X-ray photons. Therefore, numerous improvements
were developed to the backprojection algorithm
(eg Barber and Brown, 1988). Such strategies
can be shown to regularize the reconstructed images.
A good mathematical characterization is given by
Santosa and Vogelius (1990).

While many variants of Sheffield backprojection exist,
clinical and experimental EIT has largely used only
one: that which was distributed in the
Sheffield Mark I EIT system (Brown and Seagar, 1987).
The G\"ottingen Goe MF II EIT system (Hahn \etal 2001)
has also been widely used for such clinical and
experimental work, and uses a very similar reconstruction
algorithm.

This paper compares GREIT to the Sheffield Mark I
backprojection algorithm;
unfortunately, it appears that the exact formulation
of the specific algorithm version of interest has been
lost. To reconstruct it, we proceeded as follows:
since image reconstruction is linear, there exists
a mapping which can explain the transformation of
measurements into images. We obtained a large set of
measurements, $\yB_k$, and the associated
reconstructed images, $\xH_k$, and formed matrices
$\YB = [ \yB_1, \cdots, \yB_k, \cdots]$ and 
$\XB = [ \xH_1, \cdots, \xH_k, \cdots]$ by concatenation.
The Sheffield backprojection matrix, was
calculated as the least squared fit to
   $\YB \RB_{SBP} = \XB$.
In order to permit testing of EIT data against
this algorithm, it has been made available
online 
\verb$http://eidors3d.sf.net/data_contrib/db_backproj_matrix/$
under the following license:
``This matrix is copyright DC Barber and BH Brown at
  University of Sheffield. It may be used free of charge
  for research and non-commercial purposes. Commercial
  applications require a license from the University of Sheffield.''


\subsection{One step linear Gauss-Newton solvers}
\label{subsec:OSLGNS}

In this section, we elaborate the
Gauss-Newton (GN) EIT reconstruction approaches,
which have been
widely used in EIT since the late 1980's (Yorkey \etal 1987,
Cheney \etal 1990).
This approach allows use of sophisticated regularized models
of the EIT inverse problem, able to represent this
solution as a linear reconstruction matrix, which can then allow
rapid, real-time imaging.
For small variations around the reference
conductivity $\sG_r$, the relationship between $\xB$ and $\yB$ can
be linearized (giving the difference EIT forward model):
\begin{equation}\label{FM}
\yB=\JB\xB+\nB
\end{equation}
where
$\JB\in\mathbb{R}^{n_M\times n_N}$ is the Jacobian or sensitivity
matrix and $\nB\in\mathbb{R}^{n_M}$ is the measurement noise which is
assumed to be uncorrelated white Gaussian. $\JB$ is calculated from
the FEM as
$[\JB]_{ij}=\left.
     \frac{\partial[\yB]_i}{\partial[\xB]_j}
          \right|_{\sG_r}$,
and depends on the FEM, current injection patterns, the reference
conductivity, and the electrode models. This system is
underdetermined since $n_N > n_M$. 
Regularization techniques are required
in order to calculate a conductivity change
estimate, $\xH$, which is both
faithful to the measurements, $\yB$, and to
{\em a priori} constraints on a ``reasonable'' image.

The GN inverse problem seeks to
calculate a solution, $\xH$, to the EIT inverse problem
in terms of a generalized Tikhonov regularization 
expressed as the minimum of a sum of quadratic norms
\begin{equation}\label{IM}
 \|\yB-\JB\xH\|_{\Sigma_n^{-1}}^2 +
 \|\xB-\xB^\circ\|_{\Sigma_x^{-1}}^2
\end{equation}
where $\xB^\circ$ represents the expected value of element
conductivity changes, which is zero for  difference EIT.
$\SG_n\in\mathbb{R}^{n_M\times n_M}$ is
 the covariance matrix of the measurement noise $\nB$. Since
$\nB$ is uncorrelated, $\SG_n$ is a diagonal matrix with
$[\SG_n]_{i,i}=\sigma_i^2$, where $\sigma_i^2$ is the noise variance at
measurement $i$. $\SG_x\in\mathbb{R}^{n_N\times
n_N}$ is the expected image covariance.

It is worth noting that many other regularized approaches
to EIT have been used, such as the
TSVD (truncated singular value decomposition, REFS),
and
SIRT (simultaneous iterative reconstruction technique, REFS).
All linear approaches are structurally similar; however
the formulation in (\ref{IM}) is more general
as it explicitly exposes the selection of image reconstruction
parameters.



Typically,  covariance matrices
$\SG_n$ and $\SG_x$ are not calculated directly, but
are modelled heuristically from {\em a priori}
considerations as 
 $\VB = \sigma_n^{2}\SG_n$
 and
 $\PB = \sigma_x^{2}\SG_x$,
where $\sigma_n$ is the average measurement noise amplitude and
$\sigma_x$ is the {\em a priori} amplitude of conductivity change.
$\VB$ models the measurement accuracy. For uncorrelated noise,
each diagonal element is proportional to the signal to noise
ratio. For difference EIT with identical channels, $\VB=\IB$. The
regularization matrix $\PB$ may be understood to model the
likelihood of image elements and their interactions.

By solving (\ref{IM}), a linearized, one-step inverse solution is
obtained as
\begin{eqnarray}\label{GN_solution}
\xH&=&\left(
    \JB^T \frac{\VB^{-1}}{\sigma_n^2} \JB 
     +
    \frac{\PB^{-1}}{\sigma_x^2} 
    \right)^{-1}
    \JB^T \frac{\VB^{-1}}{\sigma_n^2}\yB
\nonumber \\
   &=&\left(
    \JB^T \VB^{-1} \JB + \lambda^2 \PB^{-1}
    \right)^{-1}
    \JB^T \VB^{-1} \yB
\end{eqnarray}
where parameter  $\lambda=\sigma_n/\sigma_x$ is
often called the ``regularization hyperparameter'' and
controls the trade-off
between resolution and noise attenuation in the reconstructed
image.
The matrix,
$\RB=\left(\JB^T\VB^{-1}\JB+\lambda^2\PB^{-1}\right)^{-1}\JB^T\VB^{-1}$
is the linear, one-step inverse.
In (\ref{GN_solution}), the term in the inverse is of size
$n_N\times n_N$. To save computational time, and improve inverse
accuracy and stability, matrix $\RB$ may be rewritten 
using the {\em data form}, or {\em Wiener filtering form} as:
\begin{equation}\label{opti_sol}
 \RB =\PB\JB^T
    \left(
       \JB\PB\JB^T+\lambda^2\VB
   \right)^{-1}
\end{equation}
In this formulation (\ref{opti_sol}), the size of the term in the
inverse is reduced to $n_M\times n_M$.

If image elements are assumed to be independent with identical
expected magnitude, $\PB$ becomes an identity matrix $\IB$ and
(\ref{GN_solution}) uses zeroth-order Tikhonov regularization. For
EIT, such solutions tend to push reconstructed noise toward the
boundary, since the measured data are much more sensitive to
boundary image elements. Instead, $\PB$ may be scaled with the
sensitivity of each element, so that each diagonal element $i$ 
$[\PB^{-1}]_{i,i} = \left[ \JB^T \JB
\right]_{i,i}^p$. This is the NOSER prior of Cheney \etal (1990)
for an exponent $p=1$. Many other prior matrices have been
proposed: to model image smoothness as a penalty for non-smooth
image regions, $\PB$ may be set to the discrete Laplacian filter
(Vauhkonen, 1998b), a discrete high pass Gaussian filter (Adler
and Guardo, 1996), or based on variance uniformization
constraints (Cohen-Bacrie \etal 1997).

In this paper, we compare results from the GREIT approach
with those of GN image reconstruction using the NOSER prior,
$\PB$ with $p=0.5$.
Because it is diagonal, $\PB$ can be
inverted without numerical difficulties. The choice of exponent is
a heuristic compromise between the pushing noise to the boundary
($p=0$) or to the centre ($p=1$). The NOSER algorithm is 
representative of a broad group of structurally similar 
algorithms that have been widely implemented in EIT;
it is generally understood to perform well.


\subsection{GREIT image reconstruction approach}

In this section, we elaborate the GREIT procedure
from which the reconstruction matrix, $\RB$ is
calculated.  The procedure depends on definition
of:
\begin{itemize}
\item
{\em Forward model:}
A forward model allows calculation of EIT
 measurement
data $\yB^{(k)}$ from a conductivity change distribution $\xB^{(k)}$.
The model represents the details of the body geometry,
the electrode size and contact impedance and the reference
conductivity, $\sG_r$, around which conductivity changes occur.
In this paper, the forward problem is solved using a
3D FEM using the complete electrode model
(Cheng \etal, 1989). However, it is worth noting that 
any forward model, physical or numerical, is suitable
for this procedure.

One other requirement for the forward problem model is
an estimate of the amplitude of conductivity changes $\xB^{(k)}$
encountered in the modelled EIT application. Based on the
forward model, conductivity chagnes $\xB$ result in
a measurement variance
$\sigma_{\yB}^2 = \mathrm{var}~\yB = E[ \| \yB \|^2 ]$. 
The mean $\yB$ is assumed to be zero, which implies
that positive and negative conductivity changes are
equally likely. This assumption is reasonable for a 
general purpose EIT reconstruction algorithm, although
specific EIT applications not match it exactly.
For lung images, we may calculate $\sigma_{\yB}^2$ from either
a model of the amplitude of conductivity changes due
to breathing, or from a sample of EIT measurements 
in representative applications.

\item
{\em Noise model:}
A noise model allows calculation of representative 
noise (or undesired signal) samples in EIT measurements.
For GREIT, we consider two sources of noise:
electronic measurement noise, and
electrode movement artefacts. In general, for each
source of noise $n_s$, we have noise samples $\yB^{(k)}_{n_s}$,
and an estimate of the noise amplitude variance
 $\sigma_{\nB}^2 = \mathrm{var}~\nB = E[ \| \nB \|^2 ]$. 
The mean noise is assumed to be zero in this formulation;
if necessary, a non-zero noise mean should be accounted for
by pre-processing of EIT measurements.
Note that for normalized difference EIT,
noise samples are normalized by the
reference measurements by $[\yB]_i = [\nB ]_i / [\vB_r]_i$.

\hspace{5mm}
{\em Electronic measurement noise}
is typically modelled to be uniform and Gaussian in EIT
image reconstruction. This assumption is reasonable for
a generic GREIT algorithm designed for general EIT reconstruction.
However, for a specific EIT system, measurement noise
varies with the details of the EIT hardware and patient
connection. If different gain settings are used for each
channel, these settings will alter the noise. Practical
measurements of EIT noise show that it is non-Gaussian and
variable (Hahn \etal, 2008). We recommend GREIT algorithms
be tuned to the hardware implementation. This may be 
implemented using a calibration protocol before the 
start of measurements, or by integrating a model of 
the hardware imperfections into the forward model
(Hartinger \etal, 2007).

\hspace{5mm}
{\em Electrode movement artefacts}
occur when the electrodes move, either with posture change
or with chest movements due to breathing. Several reports
have demonstrated the serious impact of such movements on
EIT images (Adler \etal, 1996, Patterson \etal, 2005,
Coulombe \etal 2005). In order to reduce the impact
of such movement on EIT reconstruction, Soleimani \etal
(2006) showed that it is possible to create an
augmented forward model based on both the conductivity change
and electrode movement, which resulted in reduced movement 
artefacts in the reconstructed images. In order
to use this capability in GREIT, we specify that a
set of ``noise'' measurements due to electrode movement 
be incorporated. This is currently implemented from
deformations of the FEM (G\'omez-Laberge and Adler, 2008),
but may be based on a calibration protocol in an
implemented system.

\begin{figure}[bhtp]
\begin{center}
\includegraphics[width=0.6\textwidth, bb=0.0in 1.7in 10in 6.0in,page=1]{figures/fig-training-set.pdf}
\caption{ \label{fig:desired_performance}
Training data for GREIT. Two sets of training data are used, based on
conductivity targets and noise or artefact sources. For each type
data, a set of training inputs (EIT measurements) and desired
outputs (reconstructed images) are calculated. Desired outputs
are based on the consensus performance measures.
}
\end{center}
\end{figure}

 
\item
{\em Desired performance metrics:}
Based on the performance metrics defined
in section \ref{sec:figmerit},
we can create a ``desired image''
$\xT^{(k)}$ corresponding to each conductivity change
$\xB^{(k)}$ in the forward model. It is possible to set
$\xT^{(k)} = \xB^{(k)}$; however, this will reduce to a standard
GN solution without tuning the algorithm to the 
performance metrics. Instead, $\xT^{(k)}$ is allowed to be
a larger circular area corresponding to the known
blurring inherent in EIT. Most EIT reconstruction algorithms
achieve better resolution near the boundary than the medium
centre; however, since we consider uniform resolution to
be a more important figure of merit than high resolution
in limited zones, we create a larger circular target, $\xT$
which can be achieved uniformly throughout the image.
We recommend, for a 16 electrode
EIT system, that the diameter of $\xH_q$
be 20\% of the medium diameter.
For noise samples, the ``desired image'' is clearly
zero; we wish for noise input to produce no output.

Corresponding to each ``desired image'', is an image
weighting $\wB^{(k)}$ which represents the weight given to 
each pixel in $\xT^{(k)}$. The weighting allows ``tuning''
of the relative importance of the  performance metrics.
For example, in order to require a low level of ringing
around each contrast, the weights are scaled  
\end{itemize}

\begin{figure}[bhtp]
\begin{center}
\includegraphics[width=0.8\textwidth, bb=0.0in 2.7in 10in 6.0in,page=1]{figures/fig-weighting.pdf}
\caption{ \label{fig:training_weighting}
Illustration of the training data and weighting.
{\em left:}   Training target ($\xB_t^{(k)}$)
{\em centre:} Reconstructed image ($\xH^{(k)}$)
{\em right:}  Desired image ($\xT^{(k)}$)
The bottom row shows the image while the top row plots
the amplitude across a row through the centre of the
simulation target. $\xT^{(k)}$ is larger than $\xB^{(k)}$
in order to establish a uniform resolution. The weighting
$\wB^{(k)}$ is illustrated on the image of $\xT^{(k)}$
by two circles. Inside the inner circle and outside the
outer circle, $\wB^{(k)}$ is larger, illustrated by the
small error bars on the upper graph. Between the circles, in
the ``transition zone'', $\wB^{(k)}$ is smaller,
 illustrated by the larger error bars.

}
\end{center}
\end{figure}


Based on the {\em forward model}, {\em noise model},
and {\em desired performance metrics}. we are
able to formulate a set of training data and
define the GREIT reconstruction matrix $\RB_{GR}$
in terms of these data. We can show that
for a linear reconstruction, this approach is 
equivalent to a generalized Tikhonov regularization
(section \ref{subsec:OSLGNS})
in which the prior distributions $\SG_x$ and $\SG_n$
correspond to the selection strategy for
the training data.
(section \ref{subsec:training_data})
The
reconstruction matrix $\RB_{GR}$ which best fits the requirements
may be expressed as minimization of an error $\epsilon^2$
\begin{equation}
\label{GREIT_norm}
\epsilon^2 = \sum_k \| \xT^{(k)} - \RB \yB^{(k)} \|_{(\WB^{(k)})^2}
    = \sum_k \| (\WB^{(k)})^2 (\xT^{(k)} - \RB \yB^{(k)} ) \|^2
\end{equation}
where $\WB^{(k)} = diag( \wB^{(k)})$, is a diagonal matrix representing
the weighting corresponding to each measurement. A 2-norm is used as it allows
a linear expression for, and faster computation of, $\RB$;
however, other norms may result in improved performance.

Based on (\ref{GREIT_norm}), we develop an expression
for $\RB = \arg\min~\epsilon^2$, by setting the
matrix derivative of $\epsilon^2$ to zero.
\begin{equation}
\frac{ d\epsilon^2 }{ d\RB_{ij} } =
-2 \sum_k (\xH^{(k)})^T (\WB^{(k)})^2 [\TG]_{i,j} \yB^{(k)}
+2 \sum_k (\yB^{(k)})^T [\TG]_{i,j} (\WB^{(k)})^2 \RB \yB_k
\end{equation}
where $[\TG]_{i,j}$ has a value of $1$ at $(i,j)$
but is zero elsewhere. Based on this expression
\begin{eqnarray}
  \sum_k (\xH^{(k)})^T (\WB^2)^{(k)} [\TG]_{i,j} \yB^{(k)}
 &=&
  \sum_k (\yB^{(k)})^T [\TG]_{i,j} (\WB^2)^{(k)} \RB \yB^{(k)}
\nonumber
\\
  \sum_k [\yB^{(k)}]_j [\wB^{(k)}]_i^2 [\xT^{(k)}]_i 
 &=&
  \sum_l [\RB]_{i,l} \sum_k [\yB^{(k)}]_l [\wB^{(k)}]_i^2 [\yB^{(k)}]_j 
\end{eqnarray}

This expression may be represented as a matrix equation
by defining matrices $\AB$ and $\BB$ where
\begin{equation}
\AB = \sum_k (\WB^{(k)})^2 \xT^{(k)} (\yB^{(k)})^T 
\end{equation}
and $\BB$ is the vertical concatenation of $n_N$
($n_M \times n_M$) block matrices $\BB_l$, where
\begin{equation}
\BB_l = \sum_k [\wB^{(k)}]_l^2 \yB^{(k)} (\yB^{(k)})^T 
\end{equation}
Based on these expressions, $\RB_{GR}$ is the
least squares minimizer of the norm 
$\|\AB_{(:)} = \RB_{(:)} \BB\|^2$ where
the notation $(:)$ row concatenation of a matrix,
Thus the GREIT reconstruction matrix, $\RB_{GR}$
is calculated
\begin{equation}
\RB_{(:)}^T = \BB (\BB^T \BB)^{-1} \AB_{(:)}^T
\end{equation}

The calculation of the GREIT reconstruction matrix
may be shown to be equivalent to a scaled generalized
Tikhonov solution of the form of section \ref{subsec:OSLGNS}.
In order to illustrate this equivalence, consider
a GREIT reconstruction algorithm formulated
with weighting $\WB$. For the training set,
$N_T$ training conductivity targets, $\yB_t^{(k)}$ are drawn from
a distribution with covariance $\SG_x$.
For this example, we consider $N_{MN}$ samples of only the electronic
measurement noise, $\yB_n^{(k)}$, which is modelled as Gaussian,
and samples are drawn from a distribution with 
covariance $\SG_n$. The reconstruction matrix $\RB$ minimizes
the norm
\begin{equation}
\| [ \XT_t, 0 ] - \RB [ \YB_t, \YB_n ] \|_{\WB^2}^2
\end{equation}
where $[\cdot,\cdot]$ represents horizontal
matrix concatenation and
$\YB_t = \frac{1}{N_T}    [ \yB_t^{(1)} \cdots \yB_t^{(N_T)} ]$,
$\XT_t = \frac{1}{N_T}    [ \xT_t^{(1)} \cdots \xT_t^{(N_T)} ]$,
and 
$\YB_n = \frac{1}{N_{MN}} [ \yB_t^{(1)} \cdots \yB_t^{(N_{MN})} ]$.
In this case, the matrix $\RB$ which minimizes the norm is 
\begin{eqnarray}
\RB &=& [ \XT_t, 0 ] [\YB_t, \YB_n]^T
     \left( [\YB_t, \YB_n] [\YB_t, \YB_n]^T \right)^{-1}
\nonumber \\
    &=& \XT_t \YB_t^T \left( \YB_t \YB_t^T + \YB_n \YB_n^T \right)^{-1}
\nonumber \\
    &=& \XT_t \YB_t^T \left( \JB \SG_x \JB^T + \SG_n \right)^{-1}
\end{eqnarray}
where samples $\YB_t$ may be approximated by the Jacobian
using the linear assumption. This formulation 
has the same generalized Tikhonov regularization as the
of (\ref{opti_sol}).


\subsection{Selection of training data}
\label{subsec:training_data}

Training data are selected as small, conductivity contrasting
targets spread randomly in the plane of the 3D model
between electrodes. Vertical offsets above and below the
plane of $0.25\times$diameter of medium are allowed.

The computational cost of calculating $\RB_{GR}$ does
not change dramatically with the size of the training data
set. The size of the matrix to be inverted is fixed at
$n_M \times n_M$. We therefore use a large number of
training samples (10000). The minimum training set 
is much smaller, because, since the model is
linear, the key requirement is that the range of 
$\RB_{GR}$ be adequately represented. This is limited
to the number of independent measurements,
$\frac{1}{2} n_E \times n_V$ (which for a 16 electrode
system is 104).

\section{Methods: forward models}

Training data for GREIT requires a forward model which
maps conductivity contrast targets $\xB_t$ to difference
measurements $\yB$. Forward models are built using
the 3D first order tetrahedral finite elements, and
solved using preconditioned linear solvers
(Polydorides and Lionheart, 2002).
Finite element models have been developed for four
different geometries: 
male and female adult chests,
a neonatal chest, and
a cylinder. The cylindrical form may be used
for phantom studies, and to image a pig thorax
(which has a roughly circular cross section).
Clearly, it is possible to design optimized
models for a given patient geometry, and this
will offer improved reconstructed images (REFS
showing improvement due to accurate models),
with the added complexity of designing
patient specific FEM meshes.
Based on our experience with time-difference
EIT, we feel that the four models provided offer
most of the accuracy of adaptive meshing.
We plan to perform tests to establish what 
image artefact levels are produced as a function
of the difference between given and the model
geometry. This will allow determination of whether
models need to be adapted 
for different patient body sizes

In the design of FEM models of the thorax. The
following issues need to be considered.
\begin{itemize}
\item
{\em Electrode placement:}
(as per discussion on electrode placement)
% COMMENT FROM INEZ I know what you mean but as it is written now it is
% not correct. I would suggest e.g. "electrodes placed at a defined chest
% height". The reason why is that an intercostal space does not lie in
% one transverse plane. Let me explain: E.g.the 5th intercostal space
% left starts at the left edge of the sternum between the 5th and 6th
% rib goes down turns arounf the chest and rises up to the corrspondnign
% vertebrum. This means that if you want to give the exact location of a
% transverse plane you need to specify 1) the intercostal space and 2)
% the exact location (i.e. in the parasternal line (directly next to
% the sternum) or in the midclavicular line (vertical line drawn at the
% middle of the clavicular bone) or midaxillar line (vertical line down
% from the axillar pit). I hope I worte this so that you were able to
% understand me...

\item
{\em Refinement of FEM mesh in electrode region:}

\item
{\em Size of FEM} in 50k elements is adequate
\item
{\em 3D FEM models are required}
\item
{\em Use of complete electrode model}
(Cheng \etal, 1989)
\item
It is necessary to simulate targets at various
\\ $-$ contrast levels
\\ $-$ background lung conductivity levels
\\ $-$ 3D (off plane) positioning
\item
\end{itemize}

Issues that have not been determined
\begin{itemize}
\item
\item
What levels of background lung conductivity should be included
\end{itemize}


The method presented in this section is general,
applying for an arbitrary number of electrodes  with 
arbitrary placement. However, the current implementation
is designed for electrodes placed in a plane around the
chest at the level of the ???



\subsection{Adult models}

Adult FEM models were generated from the visible human body dataset
(Ackerman, 1998) which provides radiological and photographic
images.  The photographic images were used to generate the surface models
from which the finite element meshes were generated.  The surface models
were created using AliasStudio (Autodesk) from the
photographic images of the thorax spaced $20~$mm apart. The method used
is similar to that used to generate models of the human head for EIT of
brain function (Tizzard \etal 2005).  Figure 1 shows an example of the
B-Spline curves generated for one slice of the male subject.

\begin{figure}[bhtp]
\begin{center}
  \includegraphics[width= 0.45\textwidth]
         {figures/adult-slice.jpg}
  \includegraphics[width= 0.35\textwidth]
         {figures/adult-model.jpg}

\caption{ \label{fig:Adultmodel}
{\em left:} One slice of the male photographic visible body dataset showing the modelled B-Spline curves of the boundary.  There are two curves, modelling the left and right sides of the boundary, tangentially blended.
{\em right:} The finished surface model of the male thorax ready for exporting to the Fe meshing software.
}
\end{center}
\end{figure}

The final set of curves are used to generate the body surfaces which are
then capped top and bottom to form a closed volume The finished surface
model is exported to the mesh-generating software which, in this case,
is I-DEAS (Integrated Design Engineering Analysis Software).
I-DEAS has the tools to define electrode geometry easily prior to meshing,
although any meshing software could be used to achieve this,
i.e. NETGEN (Schoberl 1997).
The meshes generated in I-DEAS were of similar resolution.  The
resulting FEM models are shown in Fig. \ref{fig:AdultFEM}.
For the male, the model comprised 27,170 elements and 5,548 nodes.
For the female, these figures were 27,073 and 5,346 respectively.
These data show that for the male, stretch values were between 0.160
and 0.976, with a mean and SD of 0.698 and 0.109 respectively and for
the female stretch values were between 0.198 and 0.982, with a mean and
SD of 0.702 and 0.101 respectively.

The models generated represent a high degree of accuracy of boundary
shape to a male and female subject.  One limitation is that the
source dataset from the visible
human project was generated from cryogenically frozen cadavers and
will therefore be subject to any anatomical inconsistencies arising from
the process.


\begin{figure}[bhtp]
\begin{center}
  \includegraphics[width= 0.45\textwidth]
         {figures/female_t_mdl.png}
  \includegraphics[width= 0.45\textwidth]
         {figures/male_t_mdl.png}
\caption{ \label{fig:AdultFEM}
Finite element models of Adult thoraces with electrodes
rectangular electrodes. The figure shows a wire frame mesh
of the surface finite elements. Electrodes are shown in green,
with a lighter colour electrode \#1.
{\em left:} Female subject
{\em right:} Male subject
}
\end{center}
\end{figure}


\subsection{Neonate models}
\subsection{Cylindrial models}

\begin{figure}[bhtp]
\begin{center}
  \includegraphics[width= 0.4 \textwidth, bb=0 0 444 517]
         {../../tutorial/GREIT-evaluation/simulation_3d_test02a.png}

\caption{ \label{fig:fm2}
{\em left:} FEM of the cylindrical phantom
{\em right:} FEM of the neonate chest 
}
\end{center}
\end{figure}



\section{Methods: Evaluation}

Evaluation is performed in two stages. First algorithm performance
is evaluated against the figures or merit in section
\ref{sec:figmerit}.
Next, we compare to a representative set of  clinical 
and experimental data.

\subsection{Performance figure or merit}


\section{Results}

\begin{figure}[bhtp]
\begin{center}
  \includegraphics[width= 0.2 \textwidth, bb=0 0 32 32]
         {../../tutorial/GREIT-evaluation/simulation_test_imgs/simulation_test03_1.png}
  \includegraphics[width= 0.2 \textwidth, bb=0 0 32 32]
         {../../tutorial/GREIT-evaluation/simulation_test_imgs/simulation_test03_2.png}
  \includegraphics[width= 0.2 \textwidth, bb=0 0 32 32]
         {../../tutorial/GREIT-evaluation/simulation_test_imgs/simulation_test03_4.png}
\caption{ \label{fig:rimage}
Reconstructed images. The conductivity target location is shown in green (the target is a circle, but shows as a small square in this image)
{\em left:} Sheffield Backprojection,
{\em center:} NOSER
{\em right:} GREIT
}
\end{center}
\end{figure}

\subsection{Experimental data}

\begin{figure}[bhtp]
\begin{center}
  \includegraphics[width= 0.3 \textwidth, bb=0 0 280 110]
{../../tutorial/GREIT-evaluation/simulation_test_imgs/simulation_test04_11.png}
  \includegraphics[width= 0.3 \textwidth, bb=0 0 280 110]
{../../tutorial/GREIT-evaluation/simulation_test_imgs/simulation_test04_21.png}
  \includegraphics[width= 0.3 \textwidth, bb=0 0 280 110]
{../../tutorial/GREIT-evaluation/simulation_test_imgs/simulation_test04_41.png}
\caption{ \label{fig:rimage}
Noise Figure ({\em Want: small})                        
{\em left:} Sheffield Backprojection,
{\em center:} NOSER
{\em right:} GREIT
}
\end{center}
\end{figure}

\begin{figure}[bhtp, bb=0 0 280 110]
\begin{center}
  \includegraphics[width= 0.3 \textwidth, bb=0 0 280 110]
{../../tutorial/GREIT-evaluation/simulation_test_imgs/simulation_test04_12.png}
  \includegraphics[width= 0.3 \textwidth, bb=0 0 280 110]
{../../tutorial/GREIT-evaluation/simulation_test_imgs/simulation_test04_22.png}
  \includegraphics[width= 0.3 \textwidth, bb=0 0 280 110]
{../../tutorial/GREIT-evaluation/simulation_test_imgs/simulation_test04_42.png}
\caption{ \label{fig:rimage}
Amplitude ({\em Want: uniform})
{\em left:} Sheffield Backprojection,
{\em center:} NOSER
{\em right:} GREIT
}
\end{center}
\end{figure}

\begin{figure}[bhtp, bb=0 0 280 110]
\begin{center}
  \includegraphics[width= 0.3 \textwidth, bb=0 0 280 110]
{../../tutorial/GREIT-evaluation/simulation_test_imgs/simulation_test04_13.png}
  \includegraphics[width= 0.3 \textwidth, bb=0 0 280 110]
{../../tutorial/GREIT-evaluation/simulation_test_imgs/simulation_test04_23.png}
  \includegraphics[width= 0.3 \textwidth, bb=0 0 280 110]
{../../tutorial/GREIT-evaluation/simulation_test_imgs/simulation_test04_43.png}
\caption{ \label{fig:rimage}
Position Error ({\em Want: small, uniform})
{\em left:} Sheffield Backprojection,
{\em center:} NOSER
{\em right:} GREIT
}
\end{center}
\end{figure}

\begin{figure}[bhtp, bb=0 0 280 110]
\begin{center}
  \includegraphics[width= 0.3 \textwidth, bb=0 0 280 110]
{../../tutorial/GREIT-evaluation/simulation_test_imgs/simulation_test04_14.png}
  \includegraphics[width= 0.3 \textwidth, bb=0 0 280 110]
{../../tutorial/GREIT-evaluation/simulation_test_imgs/simulation_test04_24.png}
  \includegraphics[width= 0.3 \textwidth, bb=0 0 280 110]
{../../tutorial/GREIT-evaluation/simulation_test_imgs/simulation_test04_44.png}
\caption{ \label{fig:rimage}
Point Spread Function ({\em Want: small, uniform })
{\em left:} Sheffield Backprojection,
{\em center:} NOSER
{\em right:} GREIT
}
\end{center}
\end{figure}

\begin{figure}[bhtp, bb=0 0 280 110]
\begin{center}
  \includegraphics[width= 0.3 \textwidth, bb=0 0 280 110]
{../../tutorial/GREIT-evaluation/simulation_test_imgs/simulation_test04_15.png}
  \includegraphics[width= 0.3 \textwidth, bb=0 0 280 110]
{../../tutorial/GREIT-evaluation/simulation_test_imgs/simulation_test04_25.png}
  \includegraphics[width= 0.3 \textwidth, bb=0 0 280 110]
{../../tutorial/GREIT-evaluation/simulation_test_imgs/simulation_test04_45.png}
\caption{ \label{fig:rimage}
Shape Deformation ({\em Want: small, uniform})
{\em left:} Sheffield Backprojection,
{\em center:} NOSER
{\em right:} GREIT
}
\end{center}
\end{figure}

\begin{figure}[bhtp, bb=0 0 280 110]
\begin{center}
  \includegraphics[width= 0.3 \textwidth, bb=0 0 280 110]
{../../tutorial/GREIT-evaluation/simulation_test_imgs/simulation_test04_16.png}
  \includegraphics[width= 0.3 \textwidth, bb=0 0 280 110]
{../../tutorial/GREIT-evaluation/simulation_test_imgs/simulation_test04_26.png}
  \includegraphics[width= 0.3 \textwidth, bb=0 0 280 110]
{../../tutorial/GREIT-evaluation/simulation_test_imgs/simulation_test04_46.png}
\caption{ \label{fig:rimage}
Overshoot ({\em Want: small })
{\em left:} Sheffield Backprojection,
{\em center:} NOSER
{\em right:} GREIT
}
\end{center}
\end{figure}

\section{Discussion}

$-$ Summary
\\
$-$ Review of ``recipe'' for GREIT algorithm
\\
$-$ Recommended selection of parameters
\\
$-$ Discussion of remaining steps before 
    GREIT can be released.
    


\subsection{Deliverables \& Licensing}

This paper makes the following contributions. All are
available on the internet at \verb+eidors3d.sf.net/GREIT+:
All algorithms, models and test data have been made available
under an open source license which allows
gratis commercial and non-commercial use.
\begin{itemize}
\item An algorithm (described in the section ??) for reconstruction
         of EIT images
\item Software to implement this algorithm in the Matlab/GNU Octave
         language available as part of EIDORS (Adler and Lionheart, 2006)
\\
{\em License:}
    GNU LGPL (Free Software Foundation, 2007).
   This license allows
   unrestricted use and allows the provided software to
   be provided with proprietary addons. The requirement is
   that the source code of any modifications be provided to
   the users. For a commercial vendor, this requirement would
   mean that it would be possible to review and understand any
   changes made to the software.

\item Reconstruction matrices for EIT image reconstruction for
      adult and neonate chest shapes and for a cylindrical phantom.
{\em License:} Creative Commons Attribution
   License (Creative Commons, 2007). Users are permitted
   to copy, distribute, transmit and adapt the work,
   under the conditions of attribution by citing this paper.

\item Experimental and Clinical data provided for evaluation of GREIT
{\em License:} Creative Commons Attribution
   License (Creative Commons, 2007). Users are permitted
   to copy, distribute, transmit and adapt the work,
   under the conditions of attribution by citing the
   papers as indicated.
\end{itemize}

{\em Other items to discuss}
\\
$-$ use of $p<2$



\section*{References}

\References % Harvard style references
\item[]
Adler A and Guardo R 1996 Electrical impedance tomography:
regularized imaging and contrast detection {\em IEEE Trans. Med.
Imaging} {\bf 15} 170-179

\item[]
Adler A, Guardo R and Berthiaume Y 1996 Impedance imaging of lung
ventilation: Do we need to account for chest expansion? {\em IEEE
Trans. Biomed. Eng.} {\bf 43}(4) 414-20


\item[]
Adler A and Lionheart W R B 2006
``Uses and abuses of EIDORS: An extensible software base for EIT''
{\em Physiol Meas}
27 S25--S42

\item[]
Ackerman M J  1998  
The Visible Human Project
{\em Proc. IEEE}
86 504--511


\item[]
Barber D C
Brown B H
Freeston I L, 1983
Imaging spatial distributions of resistivity using applied potential tomography
{\em Electronics Letters}
19 933--935


\item[]
Barber D C and Brown B H 1984
``Applied potential tomography'', 
{\em J Phys E: Sci Instrum}
 17 723--733

\item[]
Barber D C and Brown B H 1988 Errors in reconstruction of
resistivity images using a linear reconstruction technique {\em
Clin. Phys. Physiol. Meas.} 
9(suppl. A) 101--4

\item[]
Barber D C 1989
``A review of image reconstruction techniques for electrical
 impedance tomography''
{\em Med Phys}
16 162--169

\item[]
Brown B H and Seagar A D 1987 
``The Sheffield data collection system''
{\em Clin Phys Physiol Meas}
 8(Suppl A) 91--97

\item[]
Cheney M, Isaacson D, Newell J C, Simske S and Goble J C 1990
NOSER: an algorithm for solving the inverse conductivity problem
{\em Int.J.Imaging Syst.Technol.} 
2 66--75

\item[]
Cheng KS, Newell JC, Gisser DG, 1989
"Electrode Models for Electric Current Computed Tomography"
{\em IEEE Trans. in Biomedical Eng.}
36 918--924


\item[]
Cohen-Bacrie C  Goussard Y and Guardo R
1997
Regularized Re construction in Electrical
Impedance Tomography Using a Variance
Uniformization Constraint 
{\em IEEE Trans. Med. Imag.} 16 562-571

\item[]
Coulombe N, Gagnon H, Marquis F, Skrobik Y, Guardo R, 2005
A parametric model of the relationship between EIT and total lung volume
{\em Physiol. Meas.}
26 401--411


\item[]
Creative Commons/Science Commons 2007
``Creative Commons 3.0 Attribution License''
\verb+creativecommons.org/licenses/by/3.0/+

\item[]
GNU Lesser General Public License: Version 3, 29 June 2007
{\em Free Software Foundation, Inc.}
\verb+www.gnu.org/licenses/lgpl.html+

G\'omez-Laberge C Adler A 2008
Direct EIT Jacobian calculations for conductivity change and electrode movement
{\em Physiol. Meas.}
29 S89--S99


\item[]
Hahn G Thiel F Dudykevych T Frerichs I Gersing E
and Hellige G 2001
``Quantitative evaluation of the performance of
different electrical tomography devices''
{\em  Biomed Tech (Berl)}
46 91--95

\item[]
Hahn G Just A Dittmar J  Hellige G 2008
Systematic errors of EIT systems determined by easily-scalable
 resistive phantoms
{\em Physiol. Meas.}
 29 S163--S172


\item[]
Hansen P C 1998 {\em Rank-deficient and ill-posed problems}
SIAM Philadelphia, PA, USA

\item[]
Harris N D, Suggett A J, Barber D C and Brown B H 1988 Applied
potential tomography: a new technique for monitoring pulmonary
function {\em Clin. Phys. Physiol. Meas.} {\bf 9} 79--85

\item[]
Hartinger A E Gagnon H Guardo R 2007
Accounting for Hardware Imperfections in EIT Image
Reconstruction Algorithms.
{\em Physiol. Meas.}
28 13--S27
 
\item[]
Kotre C J 1988
``A fast approximation for the calculation of potential distributions in electrical impedance tomography''
{\em Clin Phys Physiol Meas}
9 353--361

\item[]
Lionheart W R B, Polydorides N and Borsic A 2005 Why is EIT so hard,
in {\em Electrical impedance tomography: methods, history and
applications}, Holder D S, Ed. Bristol and Philadelphia, IOP, pp.
3-4

\item[]
Lionheart W R B 2004
EIT reconstruction algorithms: pitfalls, challenges
and recent developments
{\em Physiol. Meas.}
25 125--142

\item[]
McArdle F J, Suggett A J, Brown B H, and Barber D C 1988 An
assessment of dynamic images by applied potential tomography for
monitoring pulmonary perfusion {\em Clin. Phys. Physiol. Meas.}
{\bf 9}, 87-91

\item[]
Oh S, Tang T, Sadleir R 2007
Quantitative analysis of shape change in Electrical Impedance Tomography (EIT)
in {\em IFMBE Proceedings}
17 424--427

\item[]
Polydorides N and Lionheart W R B 2002 A Matlab toolkit for
three-dimensional electrical impedance tomography: A contribution
to the Electrical Impedance and Diffuse Optical Reconstruction
Software project {\em Meas. Sci. Technol.} {\bf 13} 1871-83

\item[]
Santosa F Vogelius M 1990
Backprojection algorithm for electrical impedance imaging
{\em SIAM J. Applied Math}
50 216--243. 

\item[]
Schoberl J 1997
NETGEN: An advancing front 2D/3D-mesh generator based on abstract rules
{\em Computing and Visualization in Science}
1 41--52 

\item[]
Soleimani M, G\'omez-Laberge C and Adler A 2006 Imaging of
conductivity changes and electrode movement in EIT \PM {\bf 27}
S103--S13

\item[]
Tizzard A Horesh L Yerworth R J Holder D S Bayford R H 2005
Generating accurate finite element meshes for the forward
model of the human head in EIT
{\em Physiol. Meas.}
 26 S251--61 

\item[]
Vauhkonen M, Vad\`asz D, Karjalainen P A, Somersalo E and
Kaipio J P 1998
 Tikhonov regularization and prior information in
electrical impedance tomography
 {\em IEEE Trans Med Imaging}
17 285--93

\item[]
Yorkey T J, Webster J G and Tompkins W J 1987
Comparing reconstruction algorithms for electrical
impedance tomography
{\em IEEE Trans. Biomed. Eng}
34 843--52

\item[]
Wheeler JL, Wang W, Tang M 2002
A comparison of methods for measurement of spatial resolution in two-dimensional circular EIT images
{\em Physiol. Meas.}
23 169--176

\item[]
Zhang J and Patterson R P 2005 EIT images of ventilation: what
contributes to the resistivity changes? \PM {\bf 26} S81--S92

\endrefs

\end{document}
