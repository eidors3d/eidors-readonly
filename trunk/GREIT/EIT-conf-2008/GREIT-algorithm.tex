% $Id: GREIT-algorithm.tex,v 1.4 2008-04-13 13:29:07 aadler Exp $
\documentclass[letterpaper,twocolumn,11pt]{article}
\usepackage[margin=0.75in]{geometry}

\usepackage{graphicx}
\newcommand{\vB}{\mbox{$\mathbf{v}$}}
\newcommand{\xB}{\mbox{$\mathbf{x}$}}
\newcommand{\yB}{\mbox{$\mathbf{y}$}}
\newcommand{\JB}{\mbox{$\mathbf{J}$}}
\newcommand{\RB}{\mbox{$\mathbf{J}$}}
\newcommand{\MB}{\mbox{$\mathbf{M}$}}
%\newcommand{\SG}{\mbox{${\boldsymbol \Sigma}$}}
\newcommand{\SG}{\mbox{${\boldmath \Sigma}$}}
\newcommand{\sG}{\mbox{${\boldmath \sigma}$}}
\newcommand{\SNR}{\mbox{\small $\mathrm{SNR }$}}
\newcommand{\NF}{\mbox{\small $\mathrm{NF }$}}
\newcommand{\EIT}{\mbox{\small $\mathit{EIT }$}}
\begin{document}

%\title[GREIT: reconstruction algorithm for EIT chest images]
\title{\bf GREIT: towards a consensus EIT algorithm for lung images}

\author{Andy Adler$^{1}$,
        John Arnold$^{2}$,
        Richard Bayford$^{3}$,
        Andrea Borsic$^{4}$,
        Brian Brown$^{5}$,
        Paul Dixon$^{6}$,
\\
        Theo Faes$^{7}$,
        In\'ez Frerichs$^{8}$,
        Herv\'e Gagnon$^{9}$,
        Yvo G\"arber$^{10}$,
        William R B Lionheart$^{12}$,
\\
        Alex Hartov$^{4}$,
        G\"unter Hahn$^{13}$,
        David Holder$^{14}$,
        David Isaacson$^{15}$,
        Anjum Malik$^{16}$,
\\
        Janet Stocks$^{17}$,
        Marco Vauhkonen$^{18}$,
        Gerhard Wolf$^{2}$,
        Eung Je Woo$^{19}$,
        and many others%
       }

\date{\bf DRAFT ($Date: 2008-04-13 13:29:07 $): Please note that the
           author list is still not agreed on}
\maketitle



\begin{abstract}
Recently, Electrical Impedance Tomography (EIT) has begun to see a
significant clinical interest for monitoring of
ventilated patients.  The key capability of EIT is to
provide a real-time image distribution of ventilation in
the patient's lungs.
However, most clinical and physiological research in lung EIT
done using older and proprietary algorithms; this is
an obstacle to interpretation of EIT results because the
reconstructed images are not well characterized.
To address this issue, we propose to develop a 
consensus linear reconstruction algorithm for lung EIT.
This algorithm will be developed in three phases:
1) selection of the ``ingredients'' and evaluation 
methodology (this paper),
2) evaluation and experience with GREIT variants, and
3) consensus and definition of the GREIT algorithm.
Algorithms evalulation criteria are identified to be:
a) quantitative output for all positions,
b) reconstructed position error (low and uniform),
c) resolution (small PSF, uniform, few artefacts),
d) good noise performance,
e) low sensitivity to electrode and boundary movement,
f) good performance on clinical and experimental data.
This approach represents the consensus of a large and representative
group of experts in EIT algorithm and clinical applications.
All software and data to implement and test GREIT had been
made available under an open source license which allows free
research and commercial use.
\end{abstract}

% \noindent{\it Keywords\/}:
% Electrical Impedance Tomography,
% Lung Function Imaging,
% Image Reconstruction,

{\small
\noindent $^?$ Systems and Computer Engineering,
               Carleton University, Ottawa, Canada \\
\noindent $^?$ Health and Social Sciences,
               Middlesex University, London, UK \\
\noindent $^?$ School of Mathematics,
               University of Manchester, UK \\
               Middlesex University, London, UK \\
}

\section{Introduction}
Electrical Impedance Tomography (EIT) measures conductivity
changes within a body from current stimulation and voltage
measurement on the body surface. One of the most promising
application of EIT is for measuring the lungs, since these
are large organs which undergo large changes in conductivity
during normal functioning. Indeed, lung function measurement
was the among first physiological applications of this technology
(refs). While there are many medical imaging and instrumentation
technologies to measure the distribution of ventilation
in the lungs, EIT is unique in that it
is able to non-invasively and continuously monitor the distribution of 
ventiolation. Based on these advantages, there is significant
interest in EIT to 
monitor mechanically ventilated patients.

One limitation is that most clinical and physiological research
on lung EIT is that being done using proprietary variants of
older image reconstruction algorithms, such as the backprojection
algorithm as implemented
in the Sheffield (Brown and Seagar, 1987)
or G\"ottingen (Hahn et al, 2001) EIT systems.
This is an obstacle to clinical use of EIT because:
1) it is difficult to determine whether a given image feature is 
an physiological or artefact,
2) comparison of regional ventilation is impacted by
algorithm spatial non-uniformity and position errors,
and 
3) multi-centre studies are not possible without a 
common imaging algorithm.
Many approaches to reconstruct EIT images have been proposed,
however, it has not easy to compare among them, because 
detailed comparisons of performance have not been done. 
However, there is general consensus amongst experts
in EIT image processing of the ``ingredients'' that should
be part of a robust and high performance algorithm.

We plan to address this problem, and to develop a
consensus linear reconstruction algorithm for EIT
images of the chest.
This algorithm is named GREIT, 
the ``Graz consensus Reconstruction algorithm for EIT'',
since early discussions took place at the 2007 EIT conference
in Graz, Austria. Our aim is develop a standard which
has broad agreement from experts in the mathematical,
engineering, physiological, and clinical EIT communities.
This paper is the first step in developing GREIT:
we define the selection of ``ingredients'' in the
algorithm and the evaluation 
methodology. Subsequently, we plan to evaluate
and gain experience with GREIT ``recipes'' based on 
variants of the ingredients. This will lead to 
consensus and definition of the GREIT algorthm.

The current work is limited to the reconstruction algorithm.
We do not propose calibration tests or phantoms, standards
for image interpretation or EIT based lung parameters; 
we do not feel this sufficient experience yet to reach
consensus in these areas.
It is important to clarify that there is no financial
goal to develoment of this algorithm. Participants have
agreed not to seek patent protection on this algorithm,
and all developed algorithms, simulation
models and simulation and experimental test data used
in this algorithm will been made available as part of
the EIDORS distribution (Adler and Lionheart, 2006).

The goals identified for GREIT are for:
\\ \noindent -- 
 single and double ring electrode
configurations with Sheffield-type EIT systems, using
      adjacent current injection and measurement.
\\ \noindent -- 
 linear (real-time) reconstruction of a 2D conductivity
change image, based on a 3D forward model
\\ \noindent -- 
 quantitative reconstructions:
   given an input in transfer impedance ($\Omega$) units,
                    the output is in impedivity change ($\Omega\cdot m$))
\\ \noindent -- 
 settings for all parameters:
     any tunable parameters must have assigned
     values in the recommended algorithm.
\\ \noindent -- 
 published reconstruction matrices are to be developed for the
      reconstruction to a $32\times 32$ pixel array for
      for $16\times1$ and $8\times 2$
      electrode configurations for the following average shapes:
   a) neonate chests, 
   b) adult chests, and 
   c) cylindrical tank phantoms.
 For other shapes and electrode configurations, users will need
   to calculate the reconstruction matrices from the 
   provided source code.

As discussed, this paper clarifies the two aspects GREIT:
1) the ``ingredients'' for the algorithm, and
2) the evaluation methodology. Details are given in the
next sections

\section{``Ingredients''}

There is general agreement that the algorithm features
described in this section are represent the best approach
to linear EIT reconstruction. However, the best selections
for the details of each feature are subject to discussion
and experimentation. For example, we all agree that
regularized image reconstruction techniques are necessary,
but are not certain of the best reconstruction matrix prior.
For this reason, we use the metaphor ``ingredients'' and
``recipe''; we are certian of the ingredients, but need
to experiment to find the best recipe.

\begin{itemize}
\item[Dual Modals]

A dual reconstruction model uses a fine finite element
model (FEM) to implement the forward solution (voltages
at electrodes), and a coarses mesh for the inverse
solution. For GREIT, the forward model is a 3D FEM with
a mesh refined near the electrodes, and the reconstruction
model is square pixel mesh. Given a forward model, $F$,
which calculates a voltage measurement vector, $\vB$, from
a forward model conductivity element vector, $\sG$, we
have $\sG = F( \vB )$ 

\item[Regularized Gauss Newton Reconstruction]



\item[Image Prior with spatial correlations]

\item[Scaling for units and spatial uniformity]

\item[Hyperparameter selection method]

\end{itemize}



\begin{figure}[bhtp]
\begin{center}
  \includegraphics[width= 0.4 \textwidth]{figs/forward_model_2.eps}
  \includegraphics[width= 0.4 \textwidth]{figs/forward_model_3.eps}

\caption{ \label{fig:fm2}
{\em left:} FEM of the cylindrical phantom
{\em right:} FEM of the neonate chest 
}
\end{center}
\end{figure}
\subsection{Algorithms}
\subsection{Evaluation: model data}

Each algorithm will be evaluated against the
following criteria.

\subsubsection{Amplitude Response}
   \begin{itemize}
   \item Output amplitude is correct ($\Omega \cdot m$)
   \item the amplitude response is uniform for all radial positions.
   \end{itemize}

\subsubsection{Position Error}
   \begin{itemize}
   \item low average position error
   \item uniform position error with radial position
   \end{itemize}

\subsubsection{ Resolution}
   \begin{itemize}
   \item small average PSF size
   \item uniform PSF size with radial position
   \item no (or very little) overshoot in the PSF
  (overshoot is the negative ring around a target)
   \item regular shape (round or oval) PSF
  (backprojection, with its streaks, does badly here)
   \end{itemize}

\subsubsection{ Noise Performance}
   \begin{itemize}
   \item low average noise amplification
   \end{itemize}

\subsubsection{ Boundary shape and electrode sensitivity}
   \begin{itemize}
   \item low sensitivity to electrode movement
   \item low sensitivity to boundary distortions
         (with breathing and posture change)
   \item low sensitivity to changes in electrode contact impedance
   \end{itemize}


\subsection{Evaluation: experimental data}


 good performance on real clinical images.
     (NB, we need to develop a small database of 
          clinical images to do this)

\section{Results}
\subsection{Model data}
\subsection{Experimental data}
\subsection{Summary}
\section{Discusion}

In summary, we have developed an algorithm (GREIT)
for reconstruction of images of the thorax
from 16 EIT electrodes placed in a single plane
around the chest. This algorithm does not represent
new ideas in image reconstruction, but rather represent
a ``best of breed'' algorithm based on the experience
of a group of experts in EIT algorithms and clinical
applications.

\subsection{Deliverables}

This paper makes the following contributions. All are
available on the internet at \verb+eidors3d.sf.net/GREIT+:
\begin{itemize}
\item An algorithm (described in the section ??) for reconstruction
         of EIT images
\item Software to implement this algorithm in the Matlab/GNU Octave
         language available as part of EIDORS (Adler and Lionheart, 2006)
\item Reconstruction matrices for EIT image reconstruction for
      adult and neonate chest shapes and for a cylindrical phantom.
\end{itemize}


\subsection{Licensing}
All algorithms, models and test data have been make
available at \verb+eidors3d.sf.net/GREIT+,
under an open source license which allows
free commercial and non-commercial use. Therefore,
we licence the following aspects as follows:

{\em Algorithm and FEM models}:
   are licensed under the GNU LGPL (Free Software Foundation, 2007).
   This license allows
   unrestricted use and allows the provided software to
   be provided with proprietary addons. The requirement is
   that the source code of any modifications be provided to
   the users. For a commercial vendor, this requirement would
   mean that it would be possible to review and understand any
   changes made to the software.

{\em Reconstruction matrix, and the Experimental and Clinical Data}:
   are licensed under the Creative Commons Attribution
   License (Creative Commons, 2007). Users are permitted
   to copy to copy, distribute, transmit and adapt the work,
   under the conditions of attribution by citing the
   papers as indicated.

\subsection{Implementation }
This section describes the use of the reconstruction
matrix to generate EIT images.
The {\em input} data to the algorithm are:
\begin{itemize}
\item[]
the model type (neonate, adult, cylindrical tank) 
\item[]
   -- one dimension parameter (the chest lateral diameter) 
\item[]
   -- $13\times 16=208$ measurements in transfer impedance units
      ($\Omega = \frac{\mbox{Measured voltage}}
                      {\mbox{Stimulation current}}$)
\end{itemize}

The {\em output} from the algorithm is:
\\
   -- a pixelized image ($32\times 32$) of 
      impedivity change ($\Omega \cdot m$)


      This matrix consists of a mapping matrix $\MB$ (size $32\times 32$ of
      integers) and a reconstruction matrix $\RB$ (size $104\times P$ of
      floating point doubles).
      Given a vector $\vB$ of difference voltage measurements, where
      the $i^{\rm th}$ measurement on stimulation pattern $j$ is
      $\vB_{i+16\times(j-1)}$. A 
      The index in $\MB$ corresponding to each output pixel
      (Fig. \ref{fig:reconst_detail}) gives a row ind index into the
      reconstruction matrix 
???

\begin{figure}[bhtp]
\begin{center}
  \includegraphics[width= 0.2 \textwidth]{figs/reconstruction-detail.eps}

\caption{ \label{fig:reconst_detail}
Diagram of $32\times 32$ rasterization of the output image. Black
squares represent the output pixels and are numbered from top
to bottom, left to right.
}
\end{center}
\end{figure}



\section*{References}

\begin{itemize}
\item[]
Adler A and Lionheart W R B 2006
``Uses and abuses of EIDORS: An extensible software base for EIT''
{\em Physiol Meas}
27 S25--S42

\item[]
Barber D C and Brown B H 1984
``Applied potential tomography'', 
{\em J Phys E: Sci Instrum}
 17 723--733

\item[]
Barber D C 1989
``A review of image reconstruction techniques for electrical
 impedance tomography''
{\em Med Phys}
16 162--169

\item[]
Brown B H and Seagar A D 1987 
``The Sheffield data collection system''
{\em Clin Phys Physiol Meas}
 8(Suppl A) 91--97

\item[]
Creative Commons/Science Commons 2007
``Creative Commons 3.0 Attribution License''
\verb+creativecommons.org/licenses/by/3.0/+

\item[]
GNU Lesser General Public License: Version 3, 29 June 2007
{\em Free Software Foundation, Inc.}
\verb+www.gnu.org/licenses/lgpl.html+

\item[]
Hahn G Thiel F Dudykevych T Frerichs I Gersing E
and Hellige G 2001
``Quantitative evaluation of the performance of
different electrical tomography devices''
{\em  Biomed Tech (Berl)}
46 91--95


\item[]
Kotre C J 1988
``A fast approximation for the calculation of potential distributions in electrical impedance tomography''
{\em Clin Phys Physiol Meas}
9 353-361
\end{itemize}

\end{document}

