\documentclass[portrait,final,a0paper,fontscale=0.277]{baposter}
%\usepackage{amssymb,amsmath} % If amsmath is required
%\usepackage{times}
%\usepackage{mathptmx}      %% use fitting times fonts also in formulas
%\usepackage{titling}
%\usepackage{authblk}
%\usepackage{multicol}
\usepackage{float}
\usepackage{enumitem}
\usepackage[small,compact]{titlesec}
%\usepackage[numbers,sort&compress]{natbib}
\usepackage[T1]{fontenc}
\usepackage[UKenglish]{babel}
\usepackage[small,bf,tableposition=top,figureposition=bottom,skip=2pt]{caption}
\usepackage{booktabs, caption, supertabular, subfig} %, siunitx}
\usepackage{tikz,color}

\usepackage{calc}
\usepackage{graphicx}
\usepackage{amsmath}
\usepackage{amssymb}
\usepackage{relsize}
\usepackage{multirow}
\usepackage{rotating}
\usepackage{bm}
\usepackage{url}

\usepackage{graphicx}
\usepackage{multicol}

%\usepackage{times}
%\usepackage{helvet}
%\usepackage{bookman}
\usepackage{palatino}

\renewcommand{\captionfont}{\footnotesize}

%\graphicspath{{images/}{../images/}}
\usetikzlibrary{calc}

%%%%%%%%%%%%%%%%%%%%%%%%%%%%%%%%%%%%%%%%%%%%%%%%%%%%%%%%%%%%%%%%%%%%%%%%%%%%%%%%
%%%% Some math symbols used in the text
%%%%%%%%%%%%%%%%%%%%%%%%%%%%%%%%%%%%%%%%%%%%%%%%%%%%%%%%%%%%%%%%%%%%%%%%%%%%%%%%

%%%%%%%%%%%%%%%%%%%%%%%%%%%%%%%%%%%%%%%%%%%%%%%%%%%%%%%%%%%%%%%%%%%%%%%%%%%%%%%%
% Multicol Settings
%%%%%%%%%%%%%%%%%%%%%%%%%%%%%%%%%%%%%%%%%%%%%%%%%%%%%%%%%%%%%%%%%%%%%%%%%%%%%%%%
\setlength{\columnsep}{1.5em}
\setlength{\columnseprule}{0mm}

%%%%%%%%%%%%%%%%%%%%%%%%%%%%%%%%%%%%%%%%%%%%%%%%%%%%%%%%%%%%%%%%%%%%%%%%%%%%%%%%
% Save space in lists. Use this after the opening of the list
%%%%%%%%%%%%%%%%%%%%%%%%%%%%%%%%%%%%%%%%%%%%%%%%%%%%%%%%%%%%%%%%%%%%%%%%%%%%%%%%
\newcommand{\compresslist}{%
\setlength{\itemsep}{1pt}%
\setlength{\parskip}{0pt}%
\setlength{\parsep}{0pt}%
}

%%%%%%%%%%%%%%%%%%%%%%%%%%%%%%%%%%%%%%%%%%%%%%%%%%%%%%%%%%%%%%%%%%%%%%%%%%%%%%
%%% Begin of Document
%%%%%%%%%%%%%%%%%%%%%%%%%%%%%%%%%%%%%%%%%%%%%%%%%%%%%%%%%%%%%%%%%%%%%%%%%%%%%%

\begin{document}

%%%%%%%%%%%%%%%%%%%%%%%%%%%%%%%%%%%%%%%%%%%%%%%%%%%%%%%%%%%%%%%%%%%%%%%%%%%%%%
%%% Here starts the poster
%%%---------------------------------------------------------------------------
%%% Format it to your taste with the options
%%%%%%%%%%%%%%%%%%%%%%%%%%%%%%%%%%%%%%%%%%%%%%%%%%%%%%%%%%%%%%%%%%%%%%%%%%%%%%
% Define some colors

\definecolor{lightblue}{rgb}{0.145,0.6666,1}
%\definecolor{lightgreen}{rgb}{0.145,1,0.6666}
%\definecolor{lightblue}{rgb}{1,0.2,0.2}

%\hyphenation{resolution occlusions}
%%
\begin{poster}%
  % Poster Options
  {
  % Show grid to help with alignment
  grid=false,
  % Column spacing
  colspacing=1em,
  % Color style
  bgColorOne=white,
  bgColorTwo=white,
  borderColor=gray,
  headerColorOne=black,
  headerColorTwo=lightblue,
  headerFontColor=white,
  boxColorOne=white,
  boxColorTwo=lightblue,
  % Format of textbox
  textborder=roundedleft,
  % Format of text header
  eyecatcher=true,
  headerborder=closed,
  headerheight=0.1\textheight,
%  textfont=\sc, An example of changing the text font
  headershape=roundedright,
  headershade=shadelr,
  headerfont=\Large\bf\textsc, %Sans Serif
  textfont={\setlength{\parindent}{1.5em}},
  boxshade=plain,
%  background=shade-tb,
  background=plain,
  linewidth=2pt
  }
  % Eye Catcher
  {
%%  \includegraphics[height=5em]{images/graph_occluded.pdf}
  }
  % Title
  {\bf\textsc{EIDORS Version 3.8}}
  % Authors
  {\large \textsc{Andy Adler$^1$, Alistair Boyle$^1$, Michael G. Crabb$^2$,
        Herv\'e Gagnon$^1$, \\
        Bart{\l}omiej Grychtol$^3$,
     Nolwenn Lesparre$^4$, William R. B. Lionheart$^2$}\\
    \small
    $^1$Carleton University, Ottawa, Canada \;\;\;\;
    $^2$University of Manchester, Manchester, UK\\
    $^3$Fraunhofer Project Group for Automation in Medicine and Biotechnology PAMB, Mannheim, Germany\\
    $^4$IRSN, B.P. 17, 92262 Fontenay-aux-Roses Cedex, France
  }
  % University logo
  {% The makebox allows the title to flow into the logo, this is a hack because of the L shaped logo.
%%    \includegraphics[height=9.0em]{images/logo}
  }

%%%%%%%%%%%%%%%%%%%%%%%%%%%%%%%%%%%%%%%%%%%%%%%%%%%%%%%%%%%%%%%%%%%%%%%%%%%%%%
%%% Now define the boxes that make up the poster
%%%---------------------------------------------------------------------------
%%% Each box has a name and can be placed absolutely or relatively.
%%% The only inconvenience is that you can only specify a relative position 
%%% towards an already declared box. So if you have a box attached to the 
%%% bottom, one to the top and a third one which should be in between, you 
%%% have to specify the top and bottom boxes before you specify the middle 
%%% box.
%%%%%%%%%%%%%%%%%%%%%%%%%%%%%%%%%%%%%%%%%%%%%%%%%%%%%%%%%%%%%%%%%%%%%%%%%%%%%%
    %
    % A coloured circle useful as a bullet with an adjustably strong filling
    \newcommand{\colouredcircle}{%
    \vspace{0.6em}\noindent\tikz{\useasboundingbox (-0.2em,-0.32em) rectangle(0.2em,0.32em); \draw[draw=black,fill=lightblue,line width=0.03em] (0,0) circle(0.18em);}\hspace{1em}}

%%%%%%%%%%%%%%%%%%%%%%%%%%%%%%%%%%%%%%%%%%%%%%%%%%%%%%%%%%%%%%%%%%%%%%%%%%%%%%
  \headerbox{New Release}{name=release,column=0,row=0}{
%%%%%%%%%%%%%%%%%%%%%%%%%%%%%%%%%%%%%%%%%%%%%%%%%%%%%%%%%%%%%%%%%%%%%%%%%%%%%%
   \vspace{0.3em}
We are pleased to announce the release of EIDORS~3.8 \cite{eidors3p8}.
The software is available at \url{www.eidors.org} licensed under the GNU GPLv2 (or GPLv3).

{\centering
\includegraphics[width=0.95\columnwidth]{../mesh-eidors3p8.pdf}
}

EIDORS aims to provide free software algorithms for forward modelling
and inverse solutions
of Electrical Impedance and (to some extent) Diffusion-based Optical Tomography, in
medical, industrial and geophysical settings and to share data and promote
collaboration.
   \vspace{0.6em}
 }

%%%%%%%%%%%%%%%%%%%%%%%%%%%%%%%%%%%%%%%%%%%%%%%%%%%%%%%%%%%%%%%%%%%%%%%%%%%%%%
  \headerbox{New Features}{name=features,column=0,below=release}{
%%%%%%%%%%%%%%%%%%%%%%%%%%%%%%%%%%%%%%%%%%%%%%%%%%%%%%%%%%%%%%%%%%%%%%%%%%%%%%
Release 3.8 of EIDORS builds upon a strong foundation in reconstruction
algorithms, adding and improving a number of aspects.

%\begin{itemize}
\colouredcircle
   More stable
  iterative absolute inverse solvers (both Gauss-Newton and
  Conjugate-Gradient).
% DEPENDS ON ABS SOLVER

\colouredcircle
   Greater flexibility in parametrization choices.
% DEPENDS ON ABS SOLVER

\colouredcircle
   Native handling of unit scaling ($10^x$, $e^x$, $\ln x$, $\log_{10} x$),
 and arbitrary units.
% resistivity/conductivity, and apparent resistivity conversions.
  Natural limits for $\sigma > 0$. %~\cite{polydorides2014}.
% A framework for alternate arbitrary units.
%  Support for transparent usage in forward solutions and
%  Jacobian calculations.
% DEPENDS ON ABS SOLVER

\colouredcircle
   GREIT reconstructions in 3D

\colouredcircle
   Speed optimizations: improved Jacobian calculation, faster cache handling, and 
% Improvements to {\tt stim\_meas\_list} generated structures, resulting in
  faster forward solutions.
%  Also, the ability to
%  convert between a stimulus structure and a list of stimulus and measurement electrode pairs.
% An implementation of the parallel Jacobian calculator (pending MEX issues).

%\colouredcircle Attempts to work around broken matlab MEX file handling (cross-platform mex compile).

\colouredcircle
   Improved interfaces to NetGen and visualization.
      Compound and point electrodes in NetGen.

\colouredcircle
   Analytic calculation of dual-mesh interpolations (coarse to fine)
% as well as, Quad and hex shaped mesh elements %rev 4392

\colouredcircle
   Support for second and third order mesh elements.

\colouredcircle
   Support for Dr\"ager and Swisstom file formats
%\colouredcircle OpenEIT file format support~\cite{jones2014}.

\colouredcircle
   Expanded shape library
%\end{itemize}
   \vspace{0.6em}
   \vfil
  }

%%%%%%%%%%%%%%%%%%%%%%%%%%%%%%%%%%%%%%%%%%%%%%%%%%%%%%%%%%%%%%%%%%%%%%%%%%%%%%
  \headerbox{References}{name=references,column=0,above=bottom,below=features}{
%%%%%%%%%%%%%%%%%%%%%%%%%%%%%%%%%%%%%%%%%%%%%%%%%%%%%%%%%%%%%%%%%%%%%%%%%%%%%%
    \smaller
    \bibliographystyle{ieee}
    \renewcommand{\section}[2]{\vskip 0.05em}
\begin{thebibliography}{}
%\bibitem{polydorides2014}
%Polydorides N, Ouypornkochagorn T, McCann H. 
%{\em Proc 15th Conf}, EIT, Gananoque, Canada, 2014
\bibitem{eidors3p8}
   Adler A, Boyle A, Crabb MG et al,
   {\em EIDORS v3.8}, Zenodo,
  % \href{http://dx.doi.org/10.5281/zenodo.17559}
       {DOI:10.5281/zenodo.17559},
    2015.
\bibitem{vauhkonen2001}
   Vauhkonen M, Lionheart WRB, Heikkinen L et al,
   {\em  Physiol Meas}, 22:107--111, 2001.
\bibitem{polydorides2002phd}
   Polydorides N,
 {\em Image Reconstruction Algorithms for Soft-Field Tomography}, Ph.D. thesis,
   University of Manchester, UK, 2002.
\bibitem{polydorides2002matlab}
   Polydorides N, Lionheart WRB,
   {\em Meas Sci and Tech}, 13:1871--1883, 2002.
\bibitem{adler2006}
   Adler A, Lionheart WRB, {\em Physiol Meas}, 27:S25--S42, 2006.
\end{thebibliography}
%\input{eidors-eit2015.bbl}
   \vspace{0.6em}
  }


%%%%%%%%%%%%%%%%%%%%%%%%%%%%%%%%%%%%%%%%%%%%%%%%%%%%%%%%%%%%%%%%%%%%%%%%%%%%%%%
  \headerbox{Successes}{name=successes,column=1,row=0}{
%%%%%%%%%%%%%%%%%%%%%%%%%%%%%%%%%%%%%%%%%%%%%%%%%%%%%%%%%%%%%%%%%%%%%%%%%%%%%%%
The structure of EIDORS has been relatively stable due, in part, to some early design choices:
a modular framework and data structure,
cross-platform support, integration of meshing,
tutorials, and the contributed data repository.
These aspects, along with an open source code-base, have enabled EIDORS to
maintain research relevance.
   \vspace{0.3em}
  }
%%%%%%%%%%%%%%%%%%%%%%%%%%%%%%%%%%%%%%%%%%%%%%%%%%%%%%%%%%%%%%%%%%%%%%%%%%%%%%
\headerbox{Challenges}{name=challenges,column=1,below=successes}{
  %%%%%%%%%%%%%%%%%%%%%%%%%%%%%%%%%%%%%%%%%%%%%%%%%%%%%%%%%%%%%%%%%%%%%%%%%%%%%%
A number of challenges inherent in the implementation of EIDORS as a Matlab-based toolkit continue to recur.
There is no real Object Oriented framework: no reflection, protection, or
  automatic management of errors.
Versions of Matlab frequently vary in confounding ways that make
  maintaining a toolkit across multiple Matlab versions difficult. This is
  particularly prevalent for Windows users and ``mex" file compilation.
The data structure and subfunction complexity in EIDORS are a
  source of confusion for beginners.
% and a challenge to manage for developers.
%The expansion of EIDORS into new application domains continues to provide software development challenges at times.
%Despite this, or indeed because of this, 
Despite these challenges, EIDORS continues to develop and grow: presenting version 3.8!
   \vspace{0.3em}
  }
%%%%%%%%%%%%%%%%%%%%%%%%%%%%%%%%%%%%%%%%%%%%%%%%%%%%%%%%%%%%%%%%%%%%%%%%%%%%%%
\headerbox{Growth}{name=growth,column=1,span=2,below=challenges}{
  %%%%%%%%%%%%%%%%%%%%%%%%%%%%%%%%%%%%%%%%%%%%%%%%%%%%%%%%%%%%%%%%%%%%%%%%%%%%%%
EIDORS-related citations continue to grow. Current citation results are
shown in table~\ref{tbl:cite}.
%
The EIDORS code-base is stable with significant effort being applied to
improving test coverage, refining performance and implementing new features
(fig.~\ref{fig:loc}). In 2012, a {\tt dev} staging area was created for
contributions in progress.

\begin{figure}[H]
\centering
% trim=left botm right top
\includegraphics[width=.9\columnwidth]{../fig_loc.pdf}
\caption{\label{fig:loc}%
%graph and table combo chart, including line for test code.
  \large
  Lines of Code (LoC) in Matlab files in the EIDORS code-base vs.\ time; Total
   (red), Eidors (i.e.\ release branch, yellow), Tutorials (green), development code (blue).
   Releases are indicated by gray bars.
}
\end{figure}

\begin{table}[H]
%  \footnotesize
\centering
%From: http://amath.colorado.edu/documentation/LaTeX/reference/tables/ex1.html
\vspace{-3mm}
\caption{\large\label{tbl:cite} EIDORS Citations
 (May 2015, scholar.google.com).
}
\begin{tabular}{lcr}
  \toprule
  Paper & Date & \hspace{-2mm}Citations \\
  %{@{\extracolsep{\fill}}@{}|c|ccc|r|}
  \midrule
  \cite{vauhkonen2001} A MATLAB package for the EIDORS project {\tiny ...}  
    & 2001 & 159 \\
  \cite{polydorides2002phd} Image reconstruction algorithms for {\tiny ...}  
    & 2002 & 88 \\
  \cite{polydorides2002matlab} A Matlab toolkit for three-dimensional {\tiny ...}  
    & 2002 & 293 \\
  \cite{adler2006} Uses and abuses of {EIDORS}: An extensible {\tiny ...} 
    & 2006 & 184 \\
  \bottomrule
\end{tabular}
\vspace{-1em}
\end{table}
   \vspace{0.3em}
}

%%%%%%%%%%%%%%%%%%%%%%%%%%%%%%%%%%%%%%%%%%%%%%%%%%%%%%%%%%%%%%%%%%%%%%%%%%%%%%
\headerbox{Example features}{name=shapes,column=2,above=growth,bottomaligned=challenges}{
  %%%%%%%%%%%%%%%%%%%%%%%%%%%%%%%%%%%%%%%%%%%%%%%%%%%%%%%%%%%%%%%%%%%%%%%%%%%%%%
\begin{center}
\includegraphics[width=0.33\columnwidth]{../../../../htdocs/tutorial/GREIT/pig_control.png}%
\hspace{2mm}
\includegraphics[width=0.33\columnwidth]{../../../../htdocs/tutorial/GREIT/pig_injury.png}
 \\
Ventilation without/with lung injury

\includegraphics[width=0.43\columnwidth]{../../../../htdocs/tutorial/GREIT/vox_GREIT_sim_02b.png}%
\raisebox{4.5\height}{\scalebox{2}{$\Rightarrow$}}%
\includegraphics[width=0.43\columnwidth]{../../../../htdocs/tutorial/GREIT/vox_GREIT_sim_04b.png}
\\
3D GREIT Reconstruction


\vspace{ 2mm}
\includegraphics[width=0.33\columnwidth]{../../../../htdocs/tutorial/netgen/netgen_geometric_models13.png}%
\includegraphics[width=0.33\columnwidth]{../../../../htdocs/tutorial/netgen/netgen_geometric_models17.png}%
\includegraphics[width=0.33\columnwidth]{../../../../htdocs/tutorial/netgen/netgen_geometric_models31.png}%
\\
New Netgen interface

\vspace{ 2mm}
\includegraphics[height=0.20\columnwidth, trim= 0 14 0 0, clip]{../../../../htdocs/tutorial/EIDORS_basics/contact_impedance04a.png}%
\includegraphics[height=0.20\columnwidth, trim=32 14 0 0, clip]{../../../../htdocs/tutorial/EIDORS_basics/contact_impedance04b.png}%
\includegraphics[height=0.20\columnwidth, trim=32 14 0 0, clip]{../../../../htdocs/tutorial/EIDORS_basics/contact_impedance04c.png}%
\\
Streamlines near electrode vs $Z_c$
\end{center}
}

%%%%%%%%%%%%%%%%%%%%%%%%%%%%%%%%%%%%%%%%%%%%%%%%%%%%%%%%%%%%%%%%%%%%%%%%%%%%%%
\headerbox{Acknowledgements}{name=ack,column=1,span=2,below=growth,above=bottom}{
  %%%%%%%%%%%%%%%%%%%%%%%%%%%%%%%%%%%%%%%%%%%%%%%%%%%%%%%%%%%%%%%%%%%%%%%%%%%%%%
\noindent
Recent funding for EIDORS development thanks to: \\
\indent Swisstom AG, NSERC Canada, EPSRC UK, and IRSN France.
}
\end{poster}

\end{document}

